\documentclass[10pt,a4paper,oneside]{article}

\usepackage[utf8]{inputenc}
\usepackage{amssymb}
\usepackage{amsmath}

\newenvironment{exercice}[1][Exercice]{\begin{trivlist}
\item[\hskip \labelsep {\bfseries #1}]}{\end{trivlist}}



\begin{document}

\title{PCs MAT432}
\author{Mario Ynocente Castro}

\maketitle

\section{EDP d'ordre un et méthode des caractéristiques}

\begin{exercice}{1}

\begin{enumerate}
\item
Soit $T > 0$. On se donne une fonction dérivable $w : [0,T[ \to \mathbb{R}$ et une fonction continue $v : [0,T[ \to \mathbb{R}$ telles que

\[ \forall t \in [0,T[, w'(t) \leq v(t) w(t) \]

Montrer que: $\forall t \in [0,T[, w(t) \leq w(0) e^{\int_{0}^t v(s)ds}$

\textbf{Solution}

$f(t) = w(t) e^{-\int_{0}^{t} v(s)ds}$

$f'(t)= \underbrace{[w'(t) - w(t)v(t)]}_{\leq 0} \underbrace{e^{-\int_{0}^{t} v(s)ds}}_{> 0} \leq 0$

$f(t) = w(t) e^{-\int_{0}^{t} v(s)ds} \leq f(0) = w(0)$

$\boxed{w(t) \leq w(0) e^{\int_{0}^t v(s)ds}}$

\item
Soit $u \in \mathcal{C}^0([0,T[,\mathbb{R})$. On suppose qu'il existe une constante réelle $C$ et une fonction continue et positive $v : [0,T] \to \mathbb{R_+}$ telle que

\[ \forall t \in [0,T[, u(t) \leq C + \int_0^t u(s)v(s)ds \]

Montrer que: $\forall t \in [0,T[, u(t) \leq C e^{\int_0^t v(s)ds}$

\textbf{Solution}

$\underbrace{u(t)v(t)}_{w'(t)} \leq v(t) \underbrace{[C + \int_0^t u(s)v(s)ds]}_{w(t)}$

Par la partie 1: $C + \int_0^t u(s) v(s) ds \leq C e^{\int_0^t v(s)ds}$

Et comme: $u(t) \leq C + \int_0^t u(s)v(s)ds$

$\boxed{u(t) \leq C e^{\int_0^t v(s)ds}}$

\item
Soit $f \in \mathcal{C}^1([0,+\infty[ \times \mathbb{R}, \mathbb{R})$. On suppose qu'il existe des constantes réelles positives $A$ et $B$ telles que

\[ \forall(t,x) \in [0,+\infty[ \times \mathbb{R}, |f(t,x)| \leq A + B |x| \]

Montrer que les solutions (maximales) de l'équation différentielle

\[ \begin{cases}
y'(t) = f(t,y(t)) \\
y(0) = y_0
\end{cases} \]

sont définies sur $\mathbb{R}$.

\textbf{Notes}

\begin{itemize}
\item
On appelle \textbf{équation différentielle d'ordre 1} toute équation du type:

\[ y'(t) = f(t,y(t)) \ldots (E) \]

où $U$ est un ouvert de $\mathbb{R} \times E$ et $f : U \to E$ une application continue.

\item
Une \textbf{solution} de l'équation différentielle $(E)$ est un couple $(I,\varphi)$ où $I$ est un intervalle de $\mathbb{R}$, et $\varphi : I \to E$ est une application dérivable telle que:

\[ \begin{cases}
(t, \varphi(t)) \in U & \forall t \in I \\
\varphi'(t) = f(t,\varphi(t)) & \forall t \in I
\end{cases} \]

\item
Soit $(I_1,\varphi_1)$ et $(I_2,\varphi_2)$ deux solutions de l'équation différentielle $(E)$. On dit que la solution $(I,\varphi_1)$ est un \textbf{prolongement} de la solution $(I_2,\varphi_2)$ si:

\begin{itemize}
\item
$I_2 \subset I_1$
\item
$\forall t \in I_2, \varphi_2(t) = \varphi_1(t)$
\end{itemize}

\item
Une solution $(I,\varphi)$ de $(E)$ est dite \textbf{maximale} lorsqu'elle n'admet aucun prolongement autre qu'elle même.

\item
\textbf{\emph{Théorème de Cauchy-Lipschitz}}: Soient $U$ un ouvert de $\mathbb{R} \times E, f : U \to E$ une application de classe $\mathcal{C}^1$ sur $U$, $(t_0,y_0) \in U$. Il existe une \textbf{unique solution maximale} du problème de Cauchy

\[ \begin{cases}
y'(t) = f(t,y(t)) \\
y(t_0) = y_0
\end{cases} \]

De plus, cette solution maximale $(I,\varphi)$ vérifie:

\begin{itemize}
\item
son intervalle de définition $I$ est ouvert
\item
toute solution de $(C)$ est restriction de $(I,\varphi)$
\end{itemize}

\end{itemize}

\textbf{Solution}

\begin{itemize}
\item
On a l'unicité de la solution maximale par Cauchy-Lipschitz (car $f$ est $\mathcal{C}^1$)

$\exists y : [0,T[ \to \mathbb{R}$ solution maximale avec $T > 0$ (T : temps de vie)

\item
\textbf{Rappel}: Si $T < \infty$, il y a "explosion"

\[ \varlimsup_{t \to T^-} |y(t)| = +\infty \]

(pour $y$ définie sur $\mathbb{R_+} \times \mathbb{R}^d$).

\textbf{Corollaire}: s'il existe une fonction continue $g : \mathbb{R_+} \to \mathbb{R}$ telle que:

\[ t \in [0,T[ , |y(t)| \leq g(t) \]

alors $T = \infty$

\item
Par l'absurde, supposons $T < \infty$:

\begin{align*}
y(t) &= y_0 + \int_0^t f(s,y(s)) ds \\
|y(t)| &\leq |y_0| + \int_0^t |f(s,y(s))| ds \\
&\leq |y_0| + \int_0^t (A + B|y(s)|) ds \\
&\leq |y_0| + At + \int_0^t B|y(s)| ds \\
&< |y_0| + AT + \int_0^t B|y(s)| ds
\end{align*}

Par la partie 2 avec $u(t) = |y(t)|, C = |y_0| + AT, v(t) = B$:

\[ \forall 0 \leq t < T, |y(t)| < (|y_0| + AT)e^{Bt} \]

Donc il n'y a pas d'explosion et $T = \infty$. Contradiction.

\end{itemize}

\end{enumerate}

\end{exercice}

\begin{exercice}{4}

Soit $b \in \mathcal{C}^1(\mathbb{R}^d,\mathbb{R}^d)$. On suppose que toutes les solutions maximales de l'équation différentielle

\[ y'(t) = b(y(t)) \]

sont définies sur $\mathbb{R}$, et on note $Y(t,y_0)$, la valeur au temps $t$ de la solution maximale $y$ de donnée initiale $y(0) = y_0$. On admet que l'application $Y: \mathbb{R} \times \mathbb{R}^d \to \mathbb{R}^d$ est de classe $\mathcal{C}^1$ sur $\mathbb{R} \times \mathbb{R}^d$.

\begin{enumerate}

\item
Montrer que

\[ \forall (s,t) \in \mathbb{R}^2, \forall y_0 \in \mathbb{R}^d, Y(t,Y(s,y_0)) = Y(s + t,y_0) \]

En déduire que

\[ \forall (t,y_0) \in \mathbb{R} \times \mathbb{R}^d, \frac{\partial Y}{\partial t}(t,y_0) = \frac{\partial Y}{\partial y_0}(t,y_0)b(y_0) \]

où $\frac{\partial Y}{\partial y_0}(t,y_0)$ désigne la matrice jacobienne de $y_0 \mapsto Y(t,y_0)$.

\textbf{Solution}

On considère les fonctions de $t$: $y_1(t) = Y(t,Y(s,y_0))$ et $y_2(t) = Y(s + t,y_0)$

\begin{itemize}
\item
$y_1(0) = Y(0,Y(s,y_0)) = Y(s,y_0)$

$y_1'(t) = \partial_t Y(t,S,y_0) = b(Y(t,Y(s,y_0)) = b(y_1(t))$

\item
$y_2(0) = Y(s,y_0)$

$y_2'(t) = \partial_t Y(s + t,y_0) = b(Y(s + t,y_0)) = b(y_2(t))$
\end{itemize}

Donc on a 2 solutions maximales de $y' = b \circ y$ avec la même donnée initiale. Or Cauchy-Lipschitz implique l'unicité.\\

En dérivant par rapport à $s$:

\begin{itemize}
\item
à gauche: $\frac{\partial Y}{\partial y_0}(t,Y(s,y_0)) \frac{\partial Y}{\partial t}(s,y_0) = \frac{\partial Y}{\partial y_0}(t,Y(s,y_0)) b(Y(s,y_0))$

\item
à droite: $\frac{\partial Y}{\partial t}(t + s,y_0)$
\end{itemize}

Pour $s = 0$: $\frac{\partial Y}{\partial t}(t,y_0) = \frac{\partial Y}{\partial y_0}(t,y_0)b(y_0)$

\item
Soit $t \in \mathbb{R}$. En déduire que l'application

\begin{align*}
Y(t,\cdot): \mathbb{R}^d &\to \mathbb{R}^d \\
y_0 &\mapsto Y(t,y_0)
\end{align*}

réalise un $\mathcal{C}^1$-difféomorphisme de $\mathbb{R}^d$ sur $\mathbb{R}^d$. Quelle est son application réciproque?

\textbf{Solution}

Soit $Y_t = Y(t,\cdot) \in \mathcal{C}^1$

Par la partie 1:

\begin{align*}
Y_t \circ Y_s &= Y_{t + s} \\
Y_0 &= Id_{\mathbb{R}^d}
\end{align*}

Alors $Y_{-t} \circ Y_t = Y_t \circ Y_{-t} = Id_{\mathbb{R}^d}$, c'est-à-dire $\boxed{(Y_t)^{-1} = Y_{-t}}$. Or, $Y_{-t}$ est $\mathcal{C}^1$, donc $Y_t$ est un $\mathcal{C}^1$-difféomorphisme.

\item
Soit $u_0 \in \mathcal{C}^1(\mathbb{R}^d,\mathbb{R})$. On considère une solution $u \in \mathcal{C}^1(\mathbb{R} \times \mathbb{R}^d,\mathbb{R})$ de l'équation de transport, à coefficients non constants,

\begin{equation}
\frac{\partial u}{\partial t}(t,x) + b(x) \cdot \nabla u(t,x) = 0,\text{ et }u(0,x) = u_0(x), \text{ pour }t \in \mathbb{R} \text{ et } x \in \mathbb{R}^d.
\end{equation}

Montrer que

\[ \forall(t,x) \in \mathbb{R} \times \mathbb{R}^d, u(t,x) = u_0(Y(-t,x)) \]

\item
Déterminer les solutions $u \in \mathcal{C}^1(\mathbb{R} \times \mathbb{R}^d,\mathbb{R})$ de l'équation de transport $(1)$.

\item
\textit{Application}. Soit $u_0 \in \mathcal{C}^1(\mathbb{R}^d,\mathbb{R})$. Déterminer les solutions $u \in \mathcal{C}^1(\mathbb{R} \times \mathbb{R}^d,\mathbb{R})$ de l'équation de transport

\[ \frac{\partial u}{\partial t}(t,x) + x \cdot \nabla u(t,x) = 0,\text{ et }u(0,x) = u_0(x), \text{ pour }t \in \mathbb{R} \text{ et } x \in \mathbb{R}^d. \]

\item
Déterminer les solutions $u$, de classe $\mathcal{C}^1$ sur leur domaine de définition, de l'équation de transport

\[ \frac{\partial u}{\partial t}(t,x) + x^2 \cdot \nabla u(t,x) = 0,\text{ et }u(0,x) = x^2. \]

Quel est le domaine de définition de ces solutions?

\end{enumerate}

\end{exercice}

\section{Produit de convolution}

\begin{exercice}{1}
Soient $\alpha$ et $\beta$ deux nombres réels strictement positifs.

\begin{enumerate}
\item
Montrer l'existence et calculer la convolée suivante: $e^{-\alpha x} 1_{]0,+\infty[}(x) \star e^{-\beta x} 1_{]0,+\infty[}(x)$.

\textbf{Solution}

\[ \underbrace{e^{-\alpha x} 1_{]0,+\infty[}(x)}_{f \in L^1} \star \underbrace{e^{-\beta x} 1_{]0,+\infty[}(x)}_{g \in L^1} = \int_{\mathbb{R}} e^{-\alpha y} \underbrace{1_{]0,+\infty[}(y)}_{y > 0} e^{-\beta(x - y)} \underbrace{1_{]0,+\infty[}(x - y)}_{x > y} dy \]

$(f * g)(x) = 0$ si $x \leq 0$, sinon

\begin{align*}
\int_0^x e^{-\alpha x} e^{-\beta(x - y)} dy &= e^{-\beta x} \int_0^x e^{(\beta - \alpha)y} dy  = e^{-\beta x} \left[ \frac{e^{(\beta - \alpha)y}}{\beta - \alpha} \right]^x_0 \\
&= e^{-\beta x}(\frac{e^{(\beta - \alpha)x} - 1}{\beta - \alpha}) = \frac{e^{-\alpha x} - e^{-\beta x}}{\beta - \alpha}
\end{align*}

\[ \boxed{e^{-\alpha x} 1_{]0,+\infty[}(x) \star e^{-\beta x} 1_{]0,+\infty[}(x) = \frac{e^{-\alpha x} - e^{-\beta x}}{\beta - \alpha} 1_{x > 0}} \]

\item
Notons $f_\alpha(x) = e^{-x} x^{\alpha - 1} 1_{]0,+\infty[}(x)$.

\begin{itemize}
\item
Montrer que $f_\alpha$ est dans $L^1(\mathbb{R})$.

\item
Notons $B(\alpha,\beta) = \int_0^1 u^{\alpha - 1}(1-u)^{\beta - 1}$. Montrer la formule $f_\alpha \star f_\beta = B(\alpha,\beta) f_{\alpha + \beta}$.

\item
Montrer la \textit{formule des compléments}: $B(\alpha,\beta) = \frac{\Gamma(\alpha)\Gamma(\beta)}{\Gamma(\alpha + \beta)}$, où $\Gamma(\alpha) = \int_0^{+\infty} e^{-u} u^{\alpha - 1} du$.

\item
En déduire la relation $g_\alpha \star g_\beta = g_{\alpha + \beta}$ où $g_\alpha = \frac{1}{\Gamma(\alpha)}f_\alpha$.
\end{itemize}
\end{enumerate}
\end{exercice}

\end{document}