\documentclass[10pt,a4paper,oneside]{article}

\usepackage[utf8]{inputenc}
\usepackage{amssymb}
\usepackage{amsmath}

\newtheorem{theoreme}{Théorème}
\newtheorem{proposition}{Proposition}
\newtheorem{corollaire}{Corollaire}
\newtheorem{lemme}{Lemme}

\newenvironment{definition}[1][Definition]{\begin{trivlist}
\item[\hskip \labelsep {\bfseries #1}]}{\end{trivlist}}

\newenvironment{exemple}[1][Exemple]{\begin{trivlist}
\item[\hskip \labelsep {\bfseries #1}]}{\end{trivlist}}

\begin{document}

\title{Amphis MAT432}
\author{Mario Ynocente Castro}

\maketitle

\section{EDP d'ordre un et méthode des caractéristiques}

\subsection{Équation de transport (à coefficients constants)}

\subsubsection{Problème de Cauchy homogène}

\[ \begin{cases}
\partial_t u + b \cdot \nabla_x u = 0 & t \geq 0, x \in \mathbb{R}^d \\
u(t = 0) = g & x \in \mathbb{R}^d
\end{cases} \]

où $b = (b_1, \ldots, b_d) \in \mathbb{R}^d$ et $g: \mathbb{R}^d \to \mathbb{R}$ sont donnés

\begin{itemize}
\item
\textbf{Courbe caractéristique}: $\gamma_{\lambda}(t) = \lambda + bt, \lambda \in \mathbb{R}^d, t \geq 0$ qui vérifie:

\[ \begin{cases}
\gamma'_{\lambda}(t) = b \\
\gamma_{\lambda}(0) = \lambda
\end{cases} \]

\item
$z(t) = u(t,\gamma_{\lambda}(t)) = u(t,\lambda + bt)$

$z'(t) = \partial_t u(t,\lambda + bt) + b \cdot \nabla_x u(t,\lambda + bt) = 0$

$u(t,\underbrace{\lambda + bt}_{x}) = u(0,\lambda) = g(\lambda) \Rightarrow \boxed{u(t,x) = g(x - bt)}$

\end{itemize}

\subsubsection{Équation non homogène}

\[ \begin{cases}
\partial_t u + b \cdot \nabla_x u = f & t \geq 0, x \in \mathbb{R}^d \\
u(t = 0) = g
\end{cases} \]

où $b \in \mathbb{R}^d$ ainsi que $f(t,x)$ et $g(x)$ sont donnés.

\begin{itemize}
\item
$z(t) = u(t,\gamma_{\lambda}(t)) = u(t,\lambda + bt)$

$z'(t) = \partial_t u(t,\lambda + bt) + b \cdot \nabla_x u(t,\lambda + bt) = f(t,\lambda + bt)$

$z(t) = z(0) + \int_{0}^{t} f(s,\lambda + bs) ds$

$u(t,\underbrace{\lambda + bt}_{x}) = \underbrace{u(0,\lambda)}_{g(\lambda)} + \int_{0}^{t} f(s,\lambda + bs) ds$

$\boxed{u(t,x) = g(x - bt) + \int_{0}^{t} f(s,x -b(t - s)) ds}$

\item
\textbf{Principe de Duhamel}: Si l'on sait résoudre le problème de Cauchy homogène, alors on peut résoudre le problème non homogéne.

\end{itemize}

\subsection{Équation des ondes en dimension un}

\[ \begin{cases}
\partial^2_t u - \partial^2_x u = 0 & t \geq 0, x \in \mathbb{R} \\
u(t = 0) = g & x \in \mathbb{R} \\
\partial_t u(t = 0) = h & x \in \mathbb{R}
\end{cases} \]

où $g(x)$ et $h(x)$ sont des fonctions données.

\begin{itemize}
\item
L'équation est d'ordre 2 en temps ce qui conduit à prescrire une donnée initiale à la fois sur $u$ et $\partial_t u$ en $t = 0$.

\item
$(\partial^2_t - \partial^2_x)u = (\partial_t + \partial_x)(\partial_t - \partial_x)u$

\item
On pose $v(t,x) = (\partial_t u - \partial_x u)(t,x)$

\[ \begin{cases}
\partial_t v + \partial_x v = 0 & t \geq 0, x \in \mathbb{R} \\
v(t = 0,x) = h(x) - g'(x) & x \in \mathbb{R}
\end{cases} \]

Par la solution de l'équation de transport avec $d = 1$ et $b = 1$:

$v(t,x) = v(0,x - t) \Rightarrow \boxed{v(t,x) = h(x - t) - g'(x - t)}$

\item
Maintenant on doit résoudre:

\[ \begin{cases}
\partial_t u - \partial_x u = v & t \geq 0, x \in \mathbb{R} \\
u(t = 0) = g & x \in \mathbb{R}
\end{cases} \]

Par la solution de l'équation de transport avec $d = 1$ et $b = -1$:

$u(t,x) = g(x + t) + \int_{0}^t v(s,x + (t - s)) ds$

$u(t,x) = g(x + t) + \int_{0}^t (h(x + t - 2s) - g'(x + t - 2s)) ds$

$y = x + t - 2s \Rightarrow ds = -\frac{1}{2}dy$

$u(t,x) = g(x + t) - \frac{1}{2} \int_{x + t}^{x - t} (h(y) - g'(y)) dy$

$\boxed{u(t,x) = \frac{1}{2}[g(x + t) + g(x - t)] + \frac{1}{2} \int_{x - t}^{x + t} h(y)dy}$ (Formule d'Alembert)

\item
Toute fonction de la forme $u(t,x) = G(x - t) + H(x + t)$ est solution de $\partial^2_t u - \partial^2_x u = 0$.

\item
\textbf{Regularité}: Il suffit de prendre $g$ de classe $C^2$ et $h$ de classe $C^1$ pour avoir une solution classique.

\item
\textbf{Domaine de dépendence}: Pour $t_0 > 0, x_0 \in \mathbb{R}^d$, on considère le triangle

\[ \mathcal{T}_0 (t_0,x_0) = \{(t,x); 0 \leq t \leq t_0, |x - x_0| \leq t_0 - t\} \]

La valeur de la solution $u(t_0,x_0)$ au point $(t_0,x_0)$ ne dépend que de $g_{|[x_0-t_0,x_0+t_0]}$ et $h_{|[x_0-t_0,x_0+t_0]}$, les restrictions de $g$ et $h$ à l'intervalle $[x_0-t_0,x_0+t_0]$.

Commen conséquence, on voit que si $g$ et $h$ sont nulles sur $[x_0-t_0,x_0+t_0]$, alors $u(t_0,x_0) = 0$, et même dans le triangle $\mathcal{T}_0 (t_0,x_0)$.

\end{itemize}

\subsection{EDP non linéaire d'ordre un, en dimension un}

\[ \text{Équation de Hopf Burgers: } \begin{cases}
\partial_t u +u \partial_x (u) = 0 & t \geq 0, x \in \mathbb{R} \\
u(t = 0) = g & x \in \mathbb{R}
\end{cases} \]

\begin{itemize}
\item
La demarche que nous allons adpoter est générale est s'étend sans difficulté aux équations: $\partial_t u + a(u) \partial_x u$ sous des hypothèses naturelles pour $a(u)$.

\item
\textbf{Courbes caractéristiques}: comme l'équation considérée s'écrit sous la forme $\partial_t u + u \partial_x u$, on s'inspire de la section sur l'équation de transport en considérant la famille de courbes caractéristiques $\Gamma_\lambda$, d'équation $x = \gamma_{\lambda}(t)$, pour tout $\lambda \in \mathbb{R}$, où:

\[ \Gamma_\lambda : \begin{cases}
\gamma'_{\lambda}(t) = u(t,\gamma_{\lambda}(t)) \\
\gamma_{\lambda}(t = 0) = \lambda
\end{cases} \]

$z(t) = u(t,\gamma_{\lambda}(t))$

$z'(t) = \partial_t u(t,\gamma_{\lambda}(t)) + \underbrace{\gamma'_{\lambda}(t)}_{u(t,\gamma_{\lambda}(t))} \partial_x u(t,\gamma_{\lambda}(t)) = 0$

$z(t) = z(0) = u(0,\lambda) = g(\lambda) \Rightarrow \boxed{\gamma_{\lambda}(t) = \lambda + g(\lambda)t} \text{  } \boxed{u(t,\lambda + g(\lambda)t) = g(\lambda)}$

La détermination explicite de $u(t,x)$ en fonction de $t$ et $x$ dépend maintenant de la résolution de l'équation: $x = \lambda + g(\lambda) t$

\item
On pose $w(t) = w_\lambda(t) = \partial_x u(t,\gamma_{\lambda}(t))$, $w(0) = \partial_x u(0,\lambda) = g'(\lambda)$

$w'(t) = \underbrace{\partial_t \partial_x}_{\partial_x \partial_t} u(t,\gamma_{\lambda}(t)) + \gamma'_{\lambda}(t) \partial^2_x u(t,\gamma_{\lambda}(t))$

$w'(t) = [-\partial_x(u \partial_x u) + u\partial^2_x u](t,\gamma_{\lambda}(t)) = -(\partial_x u)^2 (t,\gamma_{\lambda}(t)) = -w^2(t)$

$-\frac{w'(t)}{w^2(t)} = (\frac{1}{w(t)})' = 1 \Rightarrow \frac{1}{w(t)} = t + \frac{1}{g'(\lambda)} \Rightarrow \boxed{w(t) = \frac{g'(\lambda)}{1 + g'(\lambda)t}}$

\item
Le fait que $w(t)$ soit ou non bien défini pour tout $t \geq 0$ dépend manifestement du signe de $g'(\lambda)$.

\begin{itemize}
\item
Si $\forall \lambda$, $g'(\lambda) \geq 0$, $T^{*} = +\infty$ (temps de vie de la solution).

\item
Si $\exists \lambda / g'(\lambda) < 0$, $T^{*} = -\frac{1}{\inf_{\lambda \in \mathbb{R}}\{ g'(\lambda)\} }  < +\infty$
\end{itemize}

Il apparaît d'après ce calcul qu'une solution classique $u(t,x)$ du problème ne peut exister que sur un intervalle de la forme $[0,\tau]$, où $0 \leq \tau < T^{*}$, puisque sinon $\partial_x u(t)$ serait mal définie à $T^{*}$.

\item
À $t$ fixée, on définit $h_t(\lambda) = \lambda + g(\lambda)t$

$h'(\lambda) = 1 + g'(\lambda)t > 0$ sur $[0,T^{*}[$

L'image de $h_t$ est $\mathbb{R}$ et $h_t$ es bijective.

$u(t,x) = g(h^{-1}_t(x))$

\end{itemize}


\section{Convolution, Équations de Laplace et de Poisson}

\subsection{Produit de convolution}

\begin{itemize}
\item
\begin{theoreme}
(Théorème de Fubini)

\begin{enumerate}
\item
Fonctions (mesurables) \textbf{positives}: $H: \mathbb{R}^{d_1} \times \mathbb{R}^{d_2} \to \bar{\mathbb{R}}_+ = [0,+\infty[ \bigcup \{+\infty\}$

\begin{align*}
\int_{\mathbb{R}^{d_1}} H(x,y) dx dy &= \int_{\mathbb{R}^{d_1}} (\int_{\mathbb{R}^{d_2}} H(x,y) dy) dx \\
&= \int_{\mathbb{R}^{d_2}} (\int_{\mathbb{R}^{d_1}} H(x,y) dx) dy
\end{align*}

\item
$h \in L^1(\mathbb{R}^{d_1} \times \mathbb{R}^{d_2})$

Pour presque tout $x$, $y \mapsto h(x,y)$ est dans $L^1(\mathbb{R}^{d_2})$, et $\varphi(x) = \int_{\mathbb{R}^{d_2}} h(x,y) dy \in L^1(\mathbb{R}^{d_1})$

\[ \int_{\mathbb{R}^{d_1}} h(x,y) dx dy = \int_{\mathbb{R}^{d_1}} (\int_{\mathbb{R}^{d_2}} h(x,y) dy) dx \]

\end{enumerate}

\textbf{Mode d'emploi}: on utilise $(1)$ sur $|h|$, puis si $\int_{\mathbb{R}^{d_1} \times \mathbb{R}^{d_2}} |h| < +\infty$, on utilise $(2)$.
\end{theoreme}

\item
\begin{definition}
Si $f$ et $g$ appartiennent à $L^1(\mathbb{R}^d)$, alors le produit de convolution $f \star g$ est défini par

\[ (f \star g)(x) = \int f(x - y)g(y) dy \int g(x - y) f(y) dy \]
\end{definition}

\item
\begin{theoreme}
Soit $f,g \in L^1(\mathbb{R}^d)$ et $h(x,y)= f(x - y)g(y)$

\begin{enumerate}
\item
Pour presque tout $x \in \mathbb{R^d}$, $y \mapsto h(x,y)$ est intégrable sur $\mathbb{R}^d$.

\item
$f \star g \in L^1(\mathbb{R}^d)$, $\| f \star g \|_{L^1} \leq \| f \|_{L^1} \| g \|_{L^1}$
\end{enumerate}
\end{theoreme}

\item
\textbf{Propiétés}:

\begin{enumerate}
\item
$f \star g = g \star f$

\item
$(f \star g) \star h = f \star (g \star h)$

\item
$f \in L^1, g \in L^2 \Rightarrow f \star g \in L^2, \| f \star g \|_{L^2} \leq \| f \|_{L^1} \| g \|_{L^2}$

\item
$f \in L^1, g \in C^0_b$ (ensemble des fonctions continues et bornées sur $\mathbb{R}^d$)

$\Rightarrow f \star g \in C^0_b$
\item
$f \in L^1, g \in C^1_b = \{ C^1,g \in C^0_b, \frac{\partial g}{\partial x_j} \in C^0_b \}$

$\Rightarrow f \star g \in C^1_b$ et $\frac{\partial}{\partial x_j}(f \star g) = f \star (\frac{\partial}{\partial x_j} g)$

\item
$f \in L^1_{loc}(\mathbb{R}^d) = \{f \text{ mesurable et } \forall \text{ K complet de } \mathbb{R}^d, f_{|K} \in L^1(K)\}$

$g \in C^1_c(\mathbb{R}^d) = \{ g \in C^1(\mathbb{R}^d) \text{ et } \exists R > 0 / g_{|\mathbb{R}^d \ B_{R}(0)} \equiv 0 \}$

$\Rightarrow f \star g \in L^1_{loc}(\mathbb{R}^d)$
\end{enumerate}

\item
\textbf{Elément neutre}: $\nexists f \in L^1, \forall g \in L^1, f \star g = g$

\item
\textbf{Approximation de l'identité}: 

Soit $h \in L^1(\mathbb{R}^d)$ vérifiant $\int_{\mathbb{R}^d} h = 1 (h \geq 0)$.

Posons, pour $\epsilon > 0$, $h_{\epsilon} (x)  = \epsilon^{-d} h(\frac{x}{\epsilon})$

\begin{theoreme}
Pour toute fonction $f \in L^1(\mathbb{R}^d)$ [resp. $L^2(\mathbb{R}^d)$], les fonctions $f \star h_\epsilon$ convergent vers $f$ en moyenne ($\| f \star h_\epsilon - f \|_{L^1} \to 0$) [resp. en moyenne quadratique] lorsque $\epsilon$ tend vers 0.
\end{theoreme}

Si de plus, $h \in C^{\infty}$, alors $\forall \epsilon > 0$, $f \star h_\epsilon \in C^{\infty}$ et $\| f \star h_\epsilon - f \|_{L^1} \to 0$
\end{itemize}

\subsection{Rappel d'intégration}

\begin{itemize}
\item
Soit $\varphi \geq 0$ ou $\varphi \in L^1([0,+\infty[)$ et $\rho(d)$ le volume de la boule unité $B_{\mathbb{R}^d}(0,1)$ de $\mathbb{R}^d$. On a:

\[ d \rho(d) \int_{0}^{+\infty} \varphi(r) dr = \int_{\mathbb{R}^d} |x|^{1 - d} \varphi(|x|) dx \]

Cette formule est une généralisation de formules bien connues en dimension d'espace 2 et 3:

\[ 2 \pi \int_{0}^{+\infty} \varphi(r) r dr = \int_{\mathbb{R}^2} \frac{\varphi(|x|)}{|x|} dx \text{ et } 4 \pi \int_{0}^{+\infty} \varphi(r) r^2 dr = \int_{\mathbb{R}^3} \frac{\varphi(|x|)}{|x|^2} dx \]

\item
\textbf{Application}: critére de Riemman en dimension d

\begin{align*}
|x|^\alpha \in L^1(B_1(0)) &\Leftrightarrow \int_{|x| < 1} |x|^\alpha dx < +\infty \\
&\Leftrightarrow \int_{0}^{1} \frac{r^\alpha}{r^{1 - d}} dr < +\infty \\
&\Leftrightarrow \alpha - 1 + d > -1 \Leftrightarrow \alpha > -d
\end{align*}

\begin{align*}
(1 + |x|)^\alpha \in L^1(\mathbb{R}^d) &\Leftrightarrow \int_{\mathbb{R}^d} (1 + |x|)^\alpha dx < +\infty \\
&\Leftrightarrow \int_{0}^{1} r^{\alpha - 1 + d} dr < +\infty \\
&\Leftrightarrow \alpha < -d
\end{align*}

\end{itemize}

\subsection{Équations de Laplace et de Poisson dans $\mathbb{R}^d$ ($d \geq 2$)}

\begin{itemize}
\item
\textbf{Équation de Laplace}: $\Delta u = 0$ dans $\mathbb{R}^d$

\item
\textbf{Équation de Poisson}: $-\Delta u = f$ dans $\mathbb{R}^d$
\end{itemize}

\subsubsection{Solution fondamentale du laplacien}

\begin{itemize}
\item
\textbf{L'équation de Laplace est invariante par rotation}: Pour toute matrice orthogonale $\Theta$ de taille $d \times d$, si $u(x)$ est solution, alors $u(\Theta x)$ est aussi une solution.

\begin{align*}
\partial_{x_i} u(\Theta x) &= \sum_{j = 1}^d (\partial_{x_j} u)(\Theta x) \Theta_{ji} \\
\partial_{x_i}^2 u(\Theta x) &= \sum_{j = 1}^d \sum_{k = 1}^d (\partial_{x_k} \partial_{x_j} u)(\Theta x) \Theta_{ji} \Theta_{ki} \\
\Delta u(\Theta x) &= \sum_{i = 1}^d \sum_{j = 1}^d \sum_{k = 1}^d (\partial_{x_k} \partial_{x_j} u)(\Theta x) \Theta_{ji} \Theta_{ki} \\
&= \sum_{j = 1}^d \sum_{k = 1}^d (\partial_{x_k} \partial_{x_j} u)(\Theta x) \underbrace{[\sum_{i = 1}^d \Theta_{ji} \Theta_{ki}]}_{\delta_{jk}} \\
&= \sum_{j = 1}^d (\partial^2_{x_j} u)(\Theta x) = 0
\end{align*}

\item
On cherche des solutions de la forme: $u(x) = v(|x|) = v(r)$ où $r = |x|$

\begin{align*}
\partial_{x_i} u(x) &= \frac{x_i}{|x|} v'(|x|) \\
\partial^2_{x_i} u(x) &= (\frac{1}{|x|} - \frac{x_i^2}{|x|^3}) v'(|x|) + \frac{x_i^2}{|x|^2} v''(|x|) \\
\Delta u(x) &= \boxed{\frac{d - 1}{r} v'(r) + v''(r) = 0}
\end{align*}

\item
$\frac{v''}{v'} = (ln (v'))' = -\frac{d - 1}{r} = -(d - 1)(ln r)'$

$ln(v') = -(d - 1)ln r + a \Rightarrow \boxed{v' = cr^{-(d - 1)}}$

\begin{itemize}
\item
$d = 2$: $v' = \frac{c}{r} \Rightarrow v = clnr + c_2$

\item
$d \geq 3$: $v' = cr^{-(d - 1)} \Rightarrow v = cr^{-d + 2} + c_2$
\end{itemize}

\item
On définit la \textbf{solution fondamentale du Laplacien}:
\[\Phi (x) =\begin{cases}
-\frac{1}{2 \pi} ln |x| & d = 2 \\
\frac{1}{d(d - 2)\rho(d)} |x|^{-(d - 2)} & d \geq 3
\end{cases}\]

\end{itemize}

\subsubsection{Résolution de l'équation de Poisson}

\begin{theoreme}
Soit $f$ de classe $\mathcal{C}^2$ sur $\mathbb{R}^d$, à support compact. On définit, pourt tout $x \in \mathbb{R}^d$,

\[ u(x) = \int_{\mathbb{R}^d} \Phi(x - y) f(y) dy = (\Phi \star f)(x) \]

Alors,

\begin{enumerate}
\item
La fonction $u$ est de classe $\mathcal{C}^2$ sur $\mathbb{R}^d$.

\item
$-\Delta u = f$ dans $\mathbb{R}^d$
\end{enumerate}
\end{theoreme}

\textbf{Preuve}:

\begin{itemize}
\item
Consiste à régulariser la fonction $\Phi$ à l'origine et à passer à la limite après convolution.

\item
Nous nous limitions au cas $d = 3$. Dans ce cas:

\[ \Phi(x) = \frac{1}{4 \pi |x|}, \Phi \in \mathcal{C}^\infty (\mathbb{R}^3 \setminus \{0 \}) \]

\item
On considère, pour tout $0 < \epsilon < 1$, $\Phi_\epsilon$ une approximation de $\Phi$ vérifiant:

\[\begin{cases}
\Phi_\epsilon \text{ est de clase } \mathcal{C}^2 \text{ sur } \mathbb{R}^3, \Phi_\epsilon(x) = \Phi_\epsilon(|x|) ;\\
\Phi_\epsilon(x) = \Phi(x) \text{ pour tout } |x| \geq \epsilon ;\\
\lim_{\epsilon \to 0} \| \Phi_\epsilon - \Phi \|_{L^1} = 0 ;\\
|\partial_{x_k} \Phi_\epsilon(x)|  \leq C \epsilon^{-2} \text{ pour tout } k = 1,2,3 \text{ et } |x| \leq \epsilon ;\\
|\partial_{x_j} \partial_{x_k} \Phi_\epsilon(x)| \leq C \epsilon^{-3} \text{ pour tout } j,k = 1,2,3 \text{ et } |x| \leq \epsilon
\end{cases}\]

On peut poser: $\Phi_\epsilon(x) = \alpha_\epsilon |x|^4 + \beta_\epsilon |x|^2 + \gamma_\epsilon$, pour tout $|x| \leq \epsilon$

et ajuster les paramètres $\alpha_\epsilon, \beta_\epsilon, \gamma_\epsilon$ pour que $\Phi_\epsilon$ soit de classe $\mathcal{C}^2$.
\end{itemize}

\end{document}