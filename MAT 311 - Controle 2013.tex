\documentclass[10pt,a4paper,oneside]{article}

\usepackage[utf8]{inputenc}
\usepackage{amssymb}
\usepackage{amsmath}
\usepackage{mathrsfs}

\newenvironment{exercice}[1][Exercice]{\begin{trivlist}
\item[\hskip \labelsep {\bfseries #1}]}{\end{trivlist}}

\newenvironment{solution}[1][Solution]{\begin{trivlist}
\item[\hskip \labelsep {\bfseries #1}]}{\end{trivlist}}



\begin{document}

\title{MAT311 - Contrôle 2013}
\author{Mario Ynocente Castro}

\maketitle

\begin{exercice}
Soit $a \in \mathbb{R}$ tel que $a > 1$.

\begin{enumerate}
\item
Déterminer les pôles de la fonction $z \mapsto \frac{z^2 - 1}{z(z^2 + 2iaz - 1)}$, et les résidus de cette fonction en ces points.

\item
En déduire la valeur de $\int_{0}^{2\pi} \frac{sint}{a + sint}dt$
\end{enumerate}
\end{exercice}

\begin{solution}
\begin{enumerate}
\item
$f(z) = \frac{z^2 - 1}{z(z^2 + 2iaz - 1)} = \frac{z^2 - 1}{z(z + ia + i\sqrt{a^2-1})(z + ia - i\sqrt{a^2-1})}$

On a les poles $z_0 = 0, z_1 = -ia - i\sqrt{a^2 - 1}, z_2 = -ia + i\sqrt{a^2 - 1}$.

\begin{itemize}
\item
$g_0(z) = (z - z_0)f(z) = \frac{z^2 - 1}{(z + ia + i\sqrt{a^2-1})(z + ia - i\sqrt{a^2-1})}$

$res_{z_0} = g_0(z_0) = \boxed{1}$

\item
$g_1(z) = (z - z_1)f(z) = \frac{z^2 - 1}{z(z + ia - i\sqrt{a^2-1})}$

$res_{z_1} = g_1(z_1) = \boxed{\frac{a}{\sqrt{a^2 - 1}}}$

\item
$g_0(z) = (z - z_2)f(z) = \frac{z^2 - 1}{z(z + ia + i\sqrt{a^2-1})}$

$res_{z_2} = g_2(z_2) = \boxed{\frac{-a}{\sqrt{a^2 - 1}}}$
\end{itemize}

\item
Pour $\gamma(t) = e^{it} = cost + isint, t \in [0,2\pi]$, on veut calculer $\int_{\gamma} \frac{z^2 - 1}{z(z^2 + 2iaz - 1)}$.

\begin{itemize}
\item
$\int_{\gamma} \frac{z^2 - 1}{z(z^2 + 2iaz - 1)} = \int_{\gamma} \frac{\frac{1}{2i}(z - \frac{1}{z})}{ \frac{1}{2i}(z - \frac{1}{z}) + a } \frac{dz}{z} = \int_{0}^{2\pi} \frac{sint}{sint + a} \frac{ie^{it}dt}{e^{it}}= i \int_{0}^{2\pi} \frac{sint}{sint + a} dt$

\item
Seulement $z_0$ et $z_2$ se trouvent dans $\gamma$. Donc, par la formule des résidus: $\int_{\gamma} \frac{z^2 - 1}{z(z^2 + 2iaz - 1)} = 2i\pi(res_{z_0} + res_{z_2}) = i (2\pi - \frac{2a\pi}{\sqrt{a^2-1}})$
\end{itemize}

Alors, $\int_{0}^{2\pi} \frac{sint}{sint + a} = \boxed{2\pi - \frac{2a\pi}{\sqrt{a^2-1}}}$.

\end{enumerate}
\end{solution}


\begin{exercice}
Soit $f: \mathbb{C} - \{ 0 \} \to \mathbb{C}$ une fonction holomorphe qui vérifie

\[ (*) \exists C_0 > 0, \text{ telle que } |f(z)| \leq C_0(|z| + \frac{1}{|z|}), \text{ pour tout } z \in \mathbb{C} - \{ 0 \} \]

\begin{enumerate}
\item
Montrer qu'il existe une constante $C > 0$ telle que $|f'(z)| \leq C(1 + \frac{1}{|z|^2})$, pour tout $z \in \mathbb{C} - \{ 0 \}$ (on pourra apliquer les estimées de Cauchy pour un cercle centré en z de rayon adéquat).

\item
En déduire que la fonction $g: \mathbb{C} \to \mathbb{C}$ définie par $g(0) = 0$ et $g(z) = z^3 f(z)$ pour tout $z \neq 0$ est holomorphe sur $\mathbb{C}$.

\item
Montrer que $g^{(5)} = 0$ où, pour tout $n \in \mathbb{N}$, $g^{(n)}$ désigne la dérivee n-ième de $g$ au sens complexe. En déduire l'expression de $g$.

\item
Déterminer toutes les fonctions $f$ qui sont holomorphes sur $\mathbb{C} - \{ 0 \}$ et vérifient $(*)$
\end{enumerate}

\end{exercice}

\begin{solution}

\begin{enumerate}
\item
Pour tout $z \in \mathbb{C} - \{ 0 \}$ on considère le disque $D(z,r = \frac{|z|}{2})$.

Pour tout $y$ sur le disque $\frac{|z|}{2} < |y| < \frac{3|z|}{2}$, alors:

\[ |f(y)| \leq C_0(|y| + \frac{1}{|y|}) < C_0(\frac{3}{2}|z| + \frac{2}{|z|}) < C_0(2|z| + \frac{2}{|z|}) = 2C_0(|z| + \frac{1}{|z|}) \]

On a $\overline{D}(z,\frac{|z|}{2}) \subset \mathbb{C} - \{ 0 \}$, alors par les estimées de Cauchy:

\[ \frac{|f'(z)|}{1!} = |f'(z)| \leq \frac{1}{r}\sup_{\partial D(z,r)}|f(z)| < \frac{2C_0}{r}(|z| + \frac{1}{|z|}) = 4C_0(1 + \frac{1}{|z|^2}) \]

On prend $C = 4C_0$.

\item
\begin{itemize}
\item
\underline{Méthode 1}:

$ | \frac{g(0 + z) - g(0)}{z} | = |\frac{g(z)}{z}| = |z|^2|f(z)| \leq C_0(|z|^3 + |z|) $

$ \Rightarrow \lim \limits_{z \to 0, z \neq 0} | \frac{g(0 + z) - g(0)}{z} | = 0$

\item
\underline{Méthode 2}:

Pour $z \in \mathbb{C} - \{ 0 \}$:

\begin{align}
|g'(z)| &= |3z^2f(z) + z^3f'(z)| \nonumber \\
&\leq 3|z|^2|f(z)| + |z|^3|f'(z)| \nonumber \\
&\leq 3C_0(|z|^3 + |z|) + C(|z|^3 + |z|)  \nonumber
\end{align}

$ \Rightarrow \lim \limits_{z \to 0} | g'(z) | = 0$
\end{itemize}

Donc $g$ est $\mathbb{C}$-dérivable en $0$, et aussi en $\mathbb{C} - \{ 0 \}$ car elle est un produit de fonctions holomorphes en $\mathbb{C} - \{ 0 \}$.

\item
Par les estimées de Cauchy pour $z_0 = 0$, et pour $n \geq 5$:

\begin{align}
|a_n| = \frac{|g^{(n)}(0)|}{n!} &\leq \frac{1}{r^n} \sup_{\partial D(0,r)}|g(z)| \nonumber \\
&= \frac{1}{r^n} \sup_{\partial D(0,r)}|z^3f(z)| \nonumber \\
& \leq \frac{C_0}{r^n}(r^4 + r^2) \text{ (car } |z^3f(z)| \leq C_0(|z|^4 + |z|^2) \text{ et } |z| = r) \nonumber \\
&= C_0(\frac{1}{r^{n - 4}} + \frac{1}{r^{n - 2}}) \nonumber
\end{align}

En faisant $r \to +\infty$, on obtient que dans le développement au tour de $z_0 = 0$, on a $a_n = 0$ pour $n \geq 5$.

Donc, $g$ est un polynóme de degré $\leq 4$, et en particulier $g^{(5)} = 0$.

\item
On sait que $f$ s'écrit: $f(z) = \sum_{i = -3}^{1} a_i z^i$

Par $(*)$:

\begin{align}
| \frac{a_{-3}}{z^3} + \frac{a_{-2}}{z^2} + \frac{a_{-1}}{z} + a_0 + a_1 z | &\leq C_0(|z| + \frac{1}{|z|}) \nonumber \\
| \frac{a_{-3}}{z^2} + \frac{a_{-2}}{z} + a_{-1} + a_0 z + a_1 z^2 | &\leq C_0(|z|^2 + 1) \nonumber
\end{align}

Si l'on considere les complexes dans le cercle $|z| \leq 1$, $C_0(|z|^2 + 1)$ est borné. Et en faisant $z \to 0$ on arrive à conclure que pour que la partie de gauche de l'inegalité soit bornée il faut avoir $a_{-3} = a_{-2} = 0$.

Donc $f$ a la forme $\boxed{ z \mapsto \frac{a_{-1}}{z} + a_0 + a_1 z }$.

Et les fonctions de cette forme vérifient $(*)$, puisque:

\begin{align}
|f(z)| &= | \frac{a_{-1}}{z} + a_0 + a_1 z | \nonumber \\
&\leq |a_{-1}|\frac{1}{|z|} + |a_0| + |a_1| |z| \nonumber \\
&\leq |a_{-1}|\frac{1}{|z|} + \frac{|a_0|}{2}(|z| + \frac{1}{|z|}) + |a_1| |z| \text{ (car } |z| + \frac{1}{|z|} \geq 2) \nonumber \\
&\leq (|a_{-1}| + |a_0| + |a_1|)(|z| + \frac{1}{|z|}) \nonumber
\end{align}

\end{enumerate}

\end{solution}


\begin{exercice}
Deux questions de dénombrabilité.

\begin{enumerate}
\item
Montrer que l'ensemble des suites $(x_n)_{n \geq 0}$, dont les coefficients sont à valeurs dans $\{ 0,1 \}$, n'est pas dénombrable (on pourra s'inspirer de la démonstration du fait que $[0,1]$ n'est pas dénombrable).

\item
Montrer que l'ensemble des suites $(x_n)_{n \geq 0}$, dont les coefficients sont à valeurs dans $\{ 0,1 \}$ et qui convergent, est dénombrable.
\end{enumerate}
\end{exercice}

\begin{solution}
\begin{enumerate}
\item
Si l'ensemble de ces suites était dénombrable. Soit $(x^m)_{m \geq 0}$ la suite de ces suites.

On peux construire la suite $(y_n)_{n \geq 0} = (1 - x^n_n)_{n \geq 0}$ qui est aussi dans l'ensemble. Mais, elle n'est pas inclus dans la suite $(x^m)_{m \geq 0}$ car par sa construction elle est differente de toutes les suites.

Donc, on arrive à une contradiction et l'ensemble n'est pas dénombrable.

\item
Comme les suites son convergentes, soit $(x_n)_{n \geq 0}$ une suite de l'ensemble, il existe $n_0 \geq 0$ tel que $\forall m \geq n_0, x_m = x_{n_0}$.

\begin{itemize}
\item
\underline{Méthode 1}:
On construira une bijection entre ces suites et $\mathbb{Z}$ ce qui montre que l'ensemble est dénombrable.

Si $x_{n_0} = 0$, on associe à la suite le nombre $2^{n_0} + \sum_{i = 0}^{n_0 - 1} 2^i x_i$.

Si $x_{n_0} = 1$, on associe à la suite le nombre $-(2^{n_0} + \sum_{i = 0}^{n_0 - 1} 2^i x_i) + 1$.

Ce qui est bien une bijection entre l'ensemble et $\mathbb{Z}$, et alors l'ensemble est dénombrable pouisque les entiers son dénombrables.

\item
\underline{Méthode 2}:
Pour une valeur fixe de $n_0$ l'ensemble de suites est fini, alors il est dénombrable. Donc l'ensemble des suites est dénombrable puisque l'union dénombrable d'ensembles dénombrables est dénombrable.
\end{itemize}

\end{enumerate}
\end{solution}


\begin{exercice}
Pour tout $f \in L^2([0,1];\mathbb{C})$ et pour tout $x \in [0,1]$, on définit

\[ T(f)(x) := ie^{i \pi x} \int_{[0,1]} (1_{[0,x]} - 1_{[x,1]}) e^{-i \pi t} f(t) dt \]

\begin{enumerate}
\item
Montrer que $T: L^2([0,1];\mathbb{C}) \to L^2([0,1];\mathbb{C})$ est un opérateur borné.

\item
Soit $f \in L^2([0,1];\mathbb{C})$. Montrer que $T(f)$ admet un représentant continu et que $\| T(f) \|_{L^{\infty}} \leq \| f \|_{L^2}$

\item
Soit $(f_n)_{n \geq 0}$ une suite d'élements de $L^2([0,1];\mathbb{C})$ qui converge faiblement vers $f$ dans $L^2([0,1];\mathbb{C}$.

Montrer que la suite $(T(f_n))_{n \geq 0}$ (ou plutôt la suite des répresentants continus) converge vers $T(f)$ en tout point de $[0,1]$.

\item
En déduire que $T$ es un opérateur compact.

\item
Montrer que $T$ est un opérateur hermitien.

\item
On suppose que $f \in \mathscr{C}([0,1];\mathbb{C})$. Calculer $T(f)'$, la derivée de $T(f)$, en fonction de $T(f)$ et de $f$.

\item
Déterminer les valeurs propres non nulles de $T$ et déterminer le spectre de $T$.
\end{enumerate}
\end{exercice}

\begin{solution}
\begin{enumerate}
\item
$\forall f \in L^2([0,1];\mathbb{C})$

\begin{align}
\|T(f)\|_{L^2} &= (\int_{[0,1]} |T(f)(x)|^2 dx)^{\frac{1}{2}} \nonumber \\
&= (\int_{[0,1]} ( \underbrace{|-e^{2i \pi x}|}_{1} |\int_{[0,1]} (1_{[0,x]} - 1_{[x,1]}) e^{-i \pi t} f(t) dt| )^2 dx)^{\frac{1}{2}} \nonumber \\
&\leq (\int_{[0,1]} ( \int_{[0,1]} \underbrace{|(1_{[0,x]} - 1_{[x,1]})|}_{1 p.p.} \underbrace{|e^{-i \pi t}|}_{1} |f(t)| dt )^2 dx)^{\frac{1}{2}} \nonumber \\
&= (\int_{[0,1]} ( \int_{[0,1]} |f(t)|dt )^2 dx)^{\frac{1}{2}} \nonumber
\end{align}

Par l'inégalité de Cauchy-Schwarz: $(\int_{[0,1]} |f(t)|dt)^2 \leq (\int_{[0,1]} |f(t)|^2 dt) \underbrace{(\int_{[0,1]} 1^2 dt)}_{1}$

Donc:

\begin{align}
\|T(f)\|_{L^2} &\leq (\int_{[0,1]} ( \int_{[0,1]} |f(t)|^2dt ) dx)^{\frac{1}{2}} \nonumber \\
&= (\int_{[0,1]} \| f \|^2_{L^2} dx)^{\frac{1}{2}} \nonumber \\
&= \| f \|_{L^2} \nonumber
\end{align}

Alors T est un opérateur borné.

\item

\item

\item

\item
On réecrit T comme: $T(f)(x) = \int_{[0,1]} ie^{i \pi x} e^{-i \pi t} (1_{[0,x]} - 1_{[x,1]}) f(t)dt$

Pour $K(x,t) = ie^{i \pi x} e^{-i \pi t} (1_{[0,x]}(t) - 1_{[x,1]}(t))$, onv voit T es l'opérateur à noyau $A_K$.

De plus, on a:

\begin{align}
\overline{K(t,x)} &= -ie^{-i \pi t} e^{i \pi x} (1_{[0,t]}(x) - 1_{[t,1]}(x)) \nonumber \\
&= -ie^{-i \pi t} e^{i \pi x} (1_{[x,1]}(t) - 1_{[0,x]}(t)) \nonumber \\
&= ie^{i \pi x} e^{-i \pi t} (1_{[0,x]}(t) - 1_{[x,1]}(t)) \nonumber \\
&= K(x,t) \nonumber
\end{align}

Donc $A_K$ es hermitien et $T$ est hermitien.

\item
$T(f)(x) = ie^{i \pi x}(\int_{[0,x]}e^{-i \pi t} f(t) dt - \int_{[x,1]}e^{-i \pi t} f(t) dt)$

$T(f)'(x) = i(i \pi) e^{i \pi x} (\int_{[0,x]}e^{-i \pi t} f(t) dt - \int_{[x,1]}e^{-i \pi t} f(t) dt) + i e^{i \pi x} (e^{-i \pi x}f(x) + e^{-i \pi x}f(x))$

$\boxed{T(f)'(x) = i \pi T(f)(x) + 2i f(x)}$

\item
Soit $\lambda \neq 0$ une valeur propre de $T$, $T(f)(x) = \lambda f(x)$.

Par la partie précédente: $\lambda f'(x) = i \pi \lambda f(x) + 2i f(x) = (i \pi \lambda + 2i)f(x)$

\[ f'(x) - i (\pi + \frac{2}{\lambda}) f(x) = 0 \Rightarrow f(x) = e^{i(\pi + \frac{2}{\lambda})x} f(0) \]

Pour $x = 0$, on a:

\[ \lambda f(0) = i e^0 (0 - \int_{[0,1]}e^{-i \pi t} f(t)dt) \Rightarrow \int_{[0,1]}e^{-i \pi t} f(t)dt = i \lambda f(0) \]

Et aussi, pour $x = 1$, 

\[ \lambda f(1) = i \underbrace{e^{i \pi}}_{-1} (\int_{[0,1]}e^{-i \pi t} f(t)dt - 0) \Rightarrow \int_{[0,1]}e^{-i \pi t} f(t)dt = i \lambda f(1) \Rightarrow f(0) = f(1) \]

De plus, par l'expression que nous avons trouvé pour $f(x)$ en $x = 1$:

\[ f(1) = e^{i(\pi + \frac{2}{\lambda})} f(0) \]
\[ \Rightarrow \pi + \frac{2}{\lambda} = 2 \pi k \Rightarrow \lambda = \frac{2}{\pi(2k - 1)}(k \in \mathbb{Z}) \]

Pour ces valeurs:

\begin{align}
T(f)(x) &= ie^{i \pi x}(\int_{[0,x]} e^{-i \pi t} e^{i \pi 2kt} dt - \int_{[x,1]} e^{-i \pi t} e^{i \pi 2kt} dt) f(0) \nonumber \\
&= ie^{i \pi x}([\frac{e^{i \pi (2k - 1) t}}{i \pi (2k - 1) t}]^x_0 - [\frac{e^{i \pi (2k - 1) t}}{i \pi (2k - 1) t}]^1_x) f(0) \nonumber \\
&= \frac{e^{i \pi x}}{\pi (2k - 1)}(2e^{i \pi (2k - 1) x} - \underbrace{e^0}_{1} - \underbrace{e^{i \pi (2k - 1)}}_{-1}) f(0) \nonumber \\
&= \frac{2e^{i \pi 2k x}}{\pi (2k - 1)} f(0) \nonumber \\
&= \lambda f(x) \nonumber
\end{align}

Donc, il sont bien les valeurs propres de $T$.

Comme $T$ est un opérateur compact et hermitien: $sp(T) = \{ \frac{2}{\pi (2k - 1)} \}_{k \in \mathbb{Z}} \bigcup \{ 0 \}$

\end{enumerate}
\end{solution}


\begin{exercice}
Soit $\varphi : [0,1] \to \mathbb{R}$ une fonction 1-lipschitzienne i.e.

\[ |\varphi(y) - \varphi(x)| \leq |y -x| \]

pour tous $x,y \in [0,1]$.

On note $\mathcal{E}$, l'espace des fonctions définies sur $[0,1]$, qui sont de la forme

\[ h = \sum_{j = 0}^{n - 1} c_j 1_{]x_j,x_{j + 1}[}, \]

où $n \geq 1$ est un entier, $c_0, \ldots, c_{n - 1}$ sont des réels et

\[ 0 = x_0 < x_1 < \ldots < x_n = 1 \]

est une partition de $[0,1]$ qui varient d'une fonction $h$ à une autre.

Pout tout $h \in \mathcal{E}$, définissons

\[ L(h) := \sum_{j = 0}^{n - 1} c_j (\varphi(x_{j + 1}) - \varphi(x_j)) \]

\begin{enumerate}
\item
Montrer que $L: \mathcal{E} \to \mathbb{R}$ est linéaire et que, pour tout $h \in \mathcal{E}$,

\[ |L(h)| \leq \| h \|_{L^1} \text{ et } |L(h)| \leq \| h \|_{L^2} \]

\item
Montrer qu'il existe $f \in \mathcal{L}^2([0,1];\mathbb{R})$ telle que, pour tout $h \in \mathcal{E}$,

\[ L(h) = \int_{[0,1]}f(t)h(t)dt \]

\item
En déduire que, pour tout $h \in \mathcal{E}$,

\[ \varphi(x) - \varphi(0) = \int_{[0,x]} f(t)dt \]

\item
Soit $c > 1$. On note $K := \{ x \in [0,1] : |f(x)| > c \}$, et on note $signf$ la fonction définie par:

\[ signf(x) := \left\{ 
  \begin{array}{l l l }
    +1 & \quad \text{si }f(x) > 0,\\
    0 & \quad \text{si }f(x) = 0,\\
    -1 & \quad \text{si }f(x) < 0.
  \end{array} \right.\]

Montrer qu'il existe une suite $h_n \in \mathcal{E}$ telle que:

\begin{enumerate}
\item
$|h_n| \leq 1$ sur $[0,1]$;
\item
La suite $(h_n)_{n \geq 0}$ converge vers $1_{K}signf$ dans $\mathcal{L}^1([0,1];\mathbb{R})$;
\item
La suite $(h_n)_{n \geq 0}$ converge presque partout vers $1_{K}signf$.
\end{enumerate}

\end{enumerate}
\end{exercice}

\begin{solution}
\begin{enumerate}
\item
\begin{itemize}
\item
$|h| = \sum_{j = 0}^{n - 1}|c_j| 1_{]x_j,x_{j + 1}[}$

\item
$\| h \|_{L^1} = \int_{0}^1 |h(t)|dt = \sum_{j = 0}^{n - 1} |c_j|(x_{j + 1}-x_j)$

\item
$\| h \|_{L^2} = (\int_{0}^1 |h(t)|^2 dt)^{\frac{1}{2}} = (\sum_{j = 0}^{n - 1} |c_j|^2(x_{j + 1}-x_j))^{\frac{1}{2}}$

\item
\begin{align}
|L(h)| &= |\sum_{j = 0}^{n - 1} c_j (\varphi(x_{j + 1}) - \varphi(x_j))| \nonumber \\
&\leq  \sum_{j = 0}^{n - 1} |c_j| |(\varphi(x_{j + 1}) - \varphi(x_j))| \text{ (par l'inégalité triangulaire)} \nonumber \\
&\leq \sum_{j = 0}^{n - 1} |c_j| |x_{j + 1} - x_{j}| \nonumber \text{ (car la fonction est 1-lipschitzienne)} \nonumber \\
&= \sum_{j = 0}^{n - 1} |c_j| (x_{j + 1} - x_{j}) = \| h \|_{L^1} \nonumber
\end{align}

\item
Par l'inégalité de Cauchy-Schwarz:

\[ (\sum_{j=0}^{n - 1} |c_j|(x_{j + 1} - x_j))^2 \leq (\sum_{j = 0}^{n - 1} |c_j|^2(x_{j + 1} - x_j)(\underbrace{ \sum_{j = 0}^{n - 1}(x_{j + 1}-x_j)) }_{1}\]
\[ \Rightarrow \| h \|_{L^1}^2 \leq \| h \|_{L^2}^2 \Rightarrow |L(h)| \leq \| h \|_{L^1} \leq \| h \|_{L^2} \]
\end{itemize}

\item
Par la partie 1, $L$ est linéaire continue sur $\mathcal{L}^2([0,1];\mathbb{R})$. Donc par le théorème de représentation de Riesz, il existe une fonction $f \in \mathcal{L}^2([0,1];\mathbb{R})$ unique telle que:

\[ \forall h \in \mathcal(E), L(h) = \langle f,h \rangle = \int_{[0,1]} f(t)h(t)dt \]

\item
Pour tout $x$, on choisit $h = 1 \cdot 1_{]0,x[} + 0 \cdot 1_{]x,1[}$.

\[ 1 \cdot (\varphi(x) - \varphi(0)) + 0 \cdot (\varphi(1) - \varphi(x)) = L(h) = \int_{[0,1]} f(t)h(t)dt \]

\[ \varphi(x) - \varphi(0) = \int_{[0,x]} f(t)dt \]

\item
%On réécrit la fonction comme $1_K signf = 1_K (1_{]0,+\infty[} - 1_{]-\infty,0[})$

\item

\item

\end{enumerate}
\end{solution}

\end{document}