\documentclass[10pt,a4paper,oneside]{article}

\usepackage[utf8]{inputenc}
\usepackage{amssymb}
\usepackage{amsmath}

\newtheorem{theoreme}{Théorème}
\newtheorem{proposition}{Proposition}
\newtheorem{corollaire}{Corollaire}
\newtheorem{lemme}{Lemme}

\newenvironment{definition}[1][Definition]{\begin{trivlist}
\item[\hskip \labelsep {\bfseries #1}]}{\end{trivlist}}

\newenvironment{exemple}[1][Exemple]{\begin{trivlist}
\item[\hskip \labelsep {\bfseries #1}]}{\end{trivlist}}

\begin{document}

\title{Résumé MAP311}
\author{Mario Ynocente Castro}

\maketitle

\section{Applications du Théorème de la Limite Centrale}

\begin{theoreme}
Soit $(X_n)_n$ une suite de v.a. réelles indépendantes, de même loi, de carré intégrable: $m = \mathbb{E}(X_1)$ et $\sigma^2 = \mathbb{V}ar(X_1)$. Alors

\[ \sqrt{n}(\frac{1}{n}\sum_{k = 1}^n X_k - m) \]

converge en loi (quand $n \to +\infty)$ vers une variable aléatoire de loi normale $N(0,\sigma^2).$
\end{theoreme}


\begin{corollaire}
Pour tout $a \leq b$,

\[ \lim\limits_{n \to +\infty} \mathbb{P}( a \leq \frac{\sqrt{n}}{\sigma}(\frac{1}{n}\sum_{k = 1}^n X_k - m) \leq b ) = \int_{a}^{b}  \frac{e^{ -x^2/2 }}{\sqrt{2 \pi }} dx \]
\end{corollaire}

\end{document}