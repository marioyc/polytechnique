\documentclass[10pt,a4paper,oneside]{article}

\usepackage[utf8]{inputenc}
\usepackage{amssymb}
\usepackage{amsmath}

\newtheorem{theoreme}{Théorème}
\newtheorem{proposition}{Proposition}
\newtheorem{corollaire}{Corollaire}
\newtheorem{lemme}{Lemme}

\newenvironment{definition}[1][Definition]{\begin{trivlist}
\item[\hskip \labelsep {\bfseries #1}]}{\end{trivlist}}

\newenvironment{exemple}[1][Exemple]{\begin{trivlist}
\item[\hskip \labelsep {\bfseries #1}]}{\end{trivlist}}

\begin{document}

\title{Résumé MAP311}
\author{Mario Ynocente Castro}

\maketitle

\section{Introduction aux Probabilités}

Soit $(B_i)_{i \in N}$ une partition finie ou dénombrable de $\Omega$.

\begin{itemize}
\item
$B_i \bigcap B_j = \emptyset$
\item
$\bigcup_i B_i = \Omega$
\end{itemize}

\textbf{Formule des probabilités totales}: $\mathbb{P}(A) = \sum_i \mathbb{P}(A \bigcap B_i) = \sum_i \mathbb{P}(A|B_i) \mathbb{P}(B_i)$

\textbf{Formule de Bayes}: Si $\mathbb{P}(A) > 0$,

\[ \forall i \in \mathbb{N}, \mathbb{P}(B_i|A) = \frac{\mathbb{P}(A \bigcap B_i)}{\mathbb{P}(A)} = \frac{\mathbb{P}(A|B_i)\mathbb{P}(B_i)}{\sum_j \mathbb{P}(A|B_j)\mathbb{P}(B_j)} \]

\section{Variables aléatoires discrètes}

\begin{itemize}
\item
\textbf{Fonction génératrice:} Pour $X$ variable aléatoire à valeurs dans $\mathbb{N}$, et $s \in [0,1]$,

\[ G_X(s) = \mathbb{E}(s^X) = \sum_{n} p_n s^n\text{, où }p_n = \mathbb{P}(X = n) \]

\begin{proposition}

\begin{enumerate}
\item
$G_X$ continue sur $[0,1]$, $C^\infty$ sur $[0,1[$. Elle caractérise la loi de $X$.

\item
$X \in \mathbb{L}^1 \Leftrightarrow G_X$ dérivable (à gauche) en $s = 1$, et

\[ \mathbb{E}(X) = G'_X(1) \]

\item
$X(X - 1)\ldots(X - p) \in \mathbb{L}^1 \Leftrightarrow G_X$ est $p + 1$ fois dérivable en $1$, et

\[ \mathbb{E}(X(X - 1)\ldots(X - p)) = G_X^{(p + 1)}(1) \]
\end{enumerate}

\end{proposition}

\end{itemize}

\section{Variables aléatoires réelles}

\begin{itemize}
\item
Si $\mathbb{E}(X^2) < +\infty$, alors $\mathbb{E}(|X|) < +\infty$ car $|X| \leq 1 + X^2$, et

\[ Var(X) = \mathbb{E}((X - \mathbb{E}(X))^2) = \mathbb{E}(X^2) - (\mathbb{E}(X))^2 \]
\[ Var(aX + b) = a^2Var(X) \]

\item
\textbf{Variable uniforme sur $[a,b]$:}

\[ f(x) = \left\{ 
  \begin{array}{l l }
    \frac{1}{b - a} & \quad \text{si }a \leq x \leq b\\
    0 & \quad \text{sinon.}
  \end{array} \right.\]
\[ \mathbb{E}(X) = \frac{a + b}{2}, Var(X) = \frac{(b - a)^2}{12} \]

\item
\textbf{Variable exponentielle $\mathcal{E}(\lambda)$:} Soit $\lambda > 0$

\[ f(x) = \left\{ 
  \begin{array}{l l }
    0 & \quad \text{si }x < 0\\
    \lambda e^{-\lambda x} & \quad \text{sinon.}
  \end{array} \right.\]
\[ \mathbb{E}(X) = \frac{1}{\lambda}, Var(X) = \frac{1}{\lambda^2} \]

\item
\textbf{Variable normale de loi $\mathcal{N}(m,\sigma^2)$:}

\[ f(x) = \frac{1}{\sqrt{2 \pi \sigma^2}} exp(-\frac{(x - m)^2}{2 \sigma^2}) \]
\[ \mathbb{E}(X) = m, Var(X) = \sigma^2 \]

\end{itemize}

\section{Applications du Théorème de la Limite Centrale}

\begin{theoreme}
Soit $(X_n)_n$ une suite de v.a. réelles indépendantes, de même loi, de carré intégrable: $m = \mathbb{E}(X_1)$ et $\sigma^2 = \mathbb{V}ar(X_1)$. Alors

\[ \sqrt{n}(\frac{1}{n}\sum_{k = 1}^n X_k - m) \]

converge en loi (quand $n \to +\infty)$ vers une variable aléatoire de loi normale $N(0,\sigma^2).$
\end{theoreme}


\begin{corollaire}
Pour tout $a \leq b$,

\[ \lim\limits_{n \to +\infty} \mathbb{P}( a \leq \frac{\sqrt{n}}{\sigma}(\frac{1}{n}\sum_{k = 1}^n X_k - m) \leq b ) = \int_{a}^{b}  \frac{e^{ -x^2/2 }}{\sqrt{2 \pi }} dx \]
\end{corollaire}

\end{document}