\documentclass[10pt,a4paper,oneside]{article}

\usepackage[utf8]{inputenc}
\usepackage{amssymb}
\usepackage{amsmath}

\newenvironment{exercice}[1][Exercice]{\begin{trivlist}
\item[\hskip \labelsep {\bfseries #1}]}{\end{trivlist}}

\newenvironment{solution}[1][Solution]{\begin{trivlist}
\item[\hskip \labelsep {\bfseries #1}]}{\end{trivlist}}



\begin{document}

\title{PCs MAT311}
\author{Mario Ynocente Castro}

\maketitle

\section{Espaces de Hilbert}

%%%%%%%%%%%%%%%%%%%%
\begin{exercice}

On considère dans $l^2(\mathbb{Z};\mathbb{R})$ le sous-espace $F = \{ (x_n)_{n \in \mathbb{Z}} : |x_n| \leq 1 \}$

\begin{enumerate}

\item
Montrer que $F$ est un convexe fermé

\item
Déterminer l'opérateur de projection orthogonale sur $F$.

\end{enumerate}

\begin{solution}

\begin{enumerate}

\item

\begin{itemize}
\item
\underline{F est convexe}:
Soient $(x_n)_{n \in \mathbb{Z}}, (y_n)_{n \in \mathbb{Z}} \in F$ et $t \in [0,1]$

$| tx_n + (1 - t)y_n| \leq t |x_n| + (1-t) |y_n| \leq 1$

Donc $(t x_n + (1 - t) y_n )_{n \in \mathbb{Z}} \in F$ et $F$ est convexe.

\item
\underline{F est fermé}:
Soit $x^n = (x^n_m)_{m \in \mathbb{Z}}$ tel que $x^n \rightarrow x^0$.

Donc $\sum (x^n_m - x^0_m)^2 \rightarrow 0$ et alors $x^n_m \rightarrow x^0_m$.

Comme $x \mapsto |x|$ est continue dans $\mathbb{R}$, on a $|x^0_m| \leq 1$.

Donc $(x^0_m)_{m \in \mathbb{Z}} \in F$ et $F$ est fermé.

\end{itemize}

\item
On pose

\[
 z_n =
  \begin{cases}
   x_n & \text{if } |x_n| \leq 1 \\
   \frac{x_n}{|x_n|} & \text{if } |x_n| > 1
  \end{cases}
\]

qui satisfait $(z_n)_{n \in \mathbb{Z}} \in F$.

\[
\sum |z_n|^2 = \underbrace{\sum_{|x_n| \leq 1} |x_n|^2}_{\text{converge car } (x_n)_{n \in \mathbb{Z}} \in l^2(\mathbb{Z};\mathbb{R})} + \underbrace{ \sum_{|x_n| > 1} }_{ \text{fini} }  \underbrace{ \frac{|x_n|^2}{|x_n|^2} }_{1} < +\infty
\]

Alors $(z_n)_{n \in \mathbb{Z}} \in l^2(\mathbb{Z};\mathbb{R})$.

\[
\forall (z'_n)_{n \in \mathbb{Z}} \in F, (x_n - z'_n)^2 \geq (x_n - z_n)^2 \Rightarrow \| (x_n) - (z'_n) \| \geq \| (x_n) - (z_n) \|
\]

Donc $p_{F}((x_n)_{n \in \mathbb{Z}}) = (z_n)_{n \in \mathbb{Z}}$.

\end{enumerate}

\end{solution}

\end{exercice}

\section{Tranformée de Fourier $L^2$, convergence faible et théorie spectrale}

%%%%%%%%%%%%%%%%%%%%
\begin{exercice}

Soit $H$ un espace de Hilbert, $V$ un sous-ensemble dense de $H$ et $(x_n)_{n \in \mathbb{N}}$ une suite de $H$ telle que 

\[
\forall h \in V, \lim\limits_{n \rightarrow +\infty} \langle x_n,h \rangle = \langle x,h \rangle
\]

\begin{enumerate}

\item
On suppose que la suite est bornée. Montrer qu'elle converge faiblement vers x.

\item
Donner un exemple de suite $(x_n)_{n \in \mathbb{N}}$ vérifiant la condition pour un sous-ensemble dense de H, mais ne convergeant pas faiblement vers x.

\end{enumerate}

\end{exercice}


\begin{solution}

\begin{enumerate}

\item
Soit $y \in H$ et $(y_n)_{n \in \mathbb{N}}$ une suite telle que $y_n \rightarrow y$ et $y_n \in V \forall n$.

\begin{align}
| \langle x_n,y \rangle - \langle x,y \rangle | &= | \langle x_n - x,y \rangle | \nonumber \\
&= | \langle x_n - x,y_p \rangle + \langle x_n - x,y-y_p \rangle | \nonumber \\
&\leq | \langle x_n,y_p \rangle - \langle x,y_p \rangle | + | \langle x_n - x,y - y_p \rangle| \nonumber \\
&\leq  | \langle x_n,y_p \rangle - \langle x,y_p \rangle | + \| x_n - x \| \| y - y_p \| (Cauchy - Schwarz) \nonumber
\end{align}

Comme $y_p \in V$, par l'hypothèse du problème:

\[
\exists N: \forall n \geq N, | \langle x_n,y_p \rangle - \langle x,y_p \rangle | \leq \frac{\epsilon}{2}
\]

Comme $(x_n)_{n \in \mathbb{N}}$ est bornée: $\exists M: \forall n, \| x_n - x \| \leq M$

Aussi, $\exists p: \| y - y_p \| \leq \frac{\epsilon}{2M}$

On a donc:

\[
\forall y \in H, \exists N: \forall n \geq N, | \langle x_n,y \rangle - \langle x,y \rangle | \leq \epsilon
\]

Alors $x_n \rightharpoonup x$.


\item
On prend $H = l^2(\mathbb{N})$, $x_n = ne_n$ et $V = \{ \sum_{i\geq0}a_i e_i | a_i = 0 \text{ à partir d'un certain point} \}$

$V$ est dense en $H$ et $\forall v \in V, \langle x_n,v \rangle \rightarrow 0$

Mais, pour $v = \sum_{n \geq 1}\frac{1}{n}e_n$, $ \langle x_n,v \rangle = 1 $

\end{enumerate}

\end{solution}

%%%%%%%%%%%%%%%%%%%%
\begin{exercice}

Soit H un espace de Hilbert et $(x_n)_{n\geq1}$ une suite qui converge faiblement vers 0. On souhaite démontrer que la suite $(x_n)_{n\geq1}$ est bornée. On raisonne par l'absurde et on suppose que (quitte à  extraire une sous-suite) la suite $(\|x_n\|)_{n\geq1}$ tend vers l'infini.

\begin{enumerate}

\item
Montrer que l'on peut construire une famille $(e_n)_{n\geq1}$ orthonormale et une fonction strictement croissante $\varphi : \mathbb{N}^* \rightarrow \mathbb{N}^*$ telles que, pout tout $n\geq1$,

\[
x_{\varphi(n)} \in Vect\{e_1,\ldots,e_n\}, \langle x_{\varphi(n)},e_n \rangle \geq n
\]

et, pour tout $n \geq 2$

\[
|\langle x_{\varphi(n)},e_i \rangle| \leq \frac{1}{n^2} \text{ pour } i = 1,\ldots,n-1
\]

\item Montrer que la série $\sum_{j = 1}^{+\infty} \frac{1}{j}e_j$ est convergente dans $H$, puis conclure.

\end{enumerate}

\end{exercice}


\begin{solution}

\begin{enumerate}

\item
On construit $\varphi: \mathbb{N}^* \rightarrow \mathbb{N}^*$ de la maniére suivante.

On prend $\varphi(1)$ tel que $\|x_1\| \geq 1$, et $e_1 = \frac{x_1}{\|x_1\|}$

Étant donnés $\varphi(1),\ldots,\varphi(n)$ et $e_1,\ldots,e_n$.

On dispose de $A \in \mathbb{N}$ tel que $\forall j \geq A$

\begin{itemize}

\item
Pour $i = 1,\ldots,n, |\langle x_{\varphi(n) + j},e_i \rangle| \leq \frac{1}{(n + 1)^2}$ (Par la convergence faible)

\item
$\|x_{\varphi(n) + j}\| \geq n + 2$ (Car $(\|x_n\|)_{n\geq1}$ tend vers l'infini)
\end{itemize}

On pose $\varphi(n + 1) = \varphi(n) + A$ et on obtient $e_{n + 1}$ à  partir de $x_{\varphi(n + 1)}$:

\[
e_{n+1} = \frac{ x_{\varphi(n) + 1} - P_{Vect\{ e_1,\ldots,e_n \}}(x_{\varphi(n + 1)}) }{\| x_{\varphi(n) + 1} - P_{Vect\{ e_1,\ldots,e_n \}}(x_{\varphi(n + 1)}) \|}
\]

On a bien que $x_{\varphi(n + 1)} \in Vect\{e_1,\ldots,e_n\}$, il nous manque démontrer que $\langle x_{\varphi(n + 1),e_{n + 1}} \rangle \geq n + 1$:

\[
(n+2)^2 \leq \| x_{\varphi(n + 1)} \| = \sum_{i = 1}^{n + 1} \langle x_{\varphi(n + 1)},e_i \rangle^2 \leq \sum_{i = 1}^{n} \frac{1}{(n + 1)^2} + \langle x_{\varphi(n + 1)},e_{n + 1} \rangle^2
\]

\[
|\langle x_{\varphi(n + 1)},e_{n + 1} \rangle| \geq (n + 2)^2 - \frac{n}{(n + 1)^2} \geq n + 1
\]

\item
$\| \sum_{j = 1}^{+\infty} \frac{1}{j}e_j \|^2 = \sum_{j = 1}^{+\infty} \frac{1}{j^2}$ converge dans $\mathbb{R}$.

Donc $\sum_{j = 1}^{+\infty}\frac{1}{j}e_j$ converge dans $H$.

$\langle x_{\varphi(n), \sum_{j = 1}^{+\infty} \frac{1}{j}e_j} = \sum_{j = 1}^n \frac{1}{j} \langle x_{\varphi(n)},e_j \rangle \geq 1 - \sum_{j = 1}^{n - 1}\frac{1}{jn^2} \geq 1 - \frac{n - 1}{n^2} \rightarrow 1$

On trouve une contradiction avec la convergence faible vers 0.

\end{enumerate}

\end{solution}

%%%%%%%%%%%%%%%%%%%%

\begin{exercice}

Soit $H$ un espace de Hilbert et $A: H \rightarrow H$ un opérateur borné.

\begin{itemize}

\item
Montrer que l'image par A d'une suite faiblement convergente est une suite faiblement convergente.

\item
Montrer que A est compact si et seulement si l'image par A des suites faiblement convergentes sont des suites (fortement) convergentes.

\end{itemize}

\end{exercice}


\begin{solution}

\begin{enumerate}

\item
Soit $(x_n)_{n\geq0}$ une suite qui converge faiblement vers x

Comme $A$ est borné, il existe l'adjoint $A^*$ et

\[
|\langle Ax_n,y \rangle - \langle Ax,y \rangle| = |\langle x_n,A^*y \rangle - \langle x,A^*y \rangle| \rightarrow 0
\]

Donc $Ax_n \rightharpoonup Ax$.

\item
($\Rightarrow$)

Soit $(x_n)_{n\geq0}$ une suite bornée tel que $x_n \rightharpoonup x$. De la première partie on sait que $Ax_n \rightharpoonup Ax$.

On suppose que $(Ax_n)_{n\geq0}$ ne converge pas vers $Ax$.

Alors, il existe $\epsilon > 0$, $(x_{\varphi(n)})_{n\geq0}$ tel que $\forall n \| Ax_{\varphi(n)} - Ax \| > \epsilon$

Comme $(x_n)_{n\geq0}$ est bornée, $Ax_{\varphi(n)}$ admet une sous-suite qui converge, et elle doit converger vers $Ax$. On obtient une contradiction.

Donc $Ax_n \rightarrow Ax$

($\Leftarrow$)

On a que l'image par A des suites faiblement convergentes sont des suites (fortement) convergentes.

Soit $(x_n)_{n\geq0}$ une suite bornée de H.
Il existe une sous-suite $(x_{\varphi(n)})_{n\geq0}$ qui converge faiblement.
D'où $(Ax_{\varphi(n)})_{n\geq0}$ converge fortement.

Donc $A$ est compact.

\end{enumerate}

\end{solution}

%%%%%%%%%%%%%%%%%%%%

\begin{exercice}

\begin{enumerate}

\item
Montrer que $\{ A \in \mathcal{L}(H,H), A$ est inversible $\}$ est un ouvert de $\mathcal{L}(H,H)$. (On pourra utiliser que pour tout opérateur $T$ avec $\| T \| < 1$ l'opérateur $Id-T$ est inversible.)

\item
Soit $A:H \rightarrow H$ un opérateur borné. Montrer que $sp(A)$, le spectre de $A$, est fermé dans $\mathbb{C}$

\end{enumerate}

\end{exercice}


\begin{solution}

\begin{enumerate}

\item
Soit $A \in \mathcal{L}(H,H)$ inversible et $B \in \mathcal{L}(H,H)$ tel que $\| B \| < \frac{1}{A^{-1}}$

$A + B = A(Id + A^{-1}B)$

$\| A^{-1}B \| < 1$, alors $A+B$ est inversible.

\item
$\phi: \lambda \in \mathbb{C} \rightarrow A-\lambda Id \in \mathcal{L}(H,H)$ est continue.


\end{enumerate}

\end{solution}

%%%%%%%%%%%%%%%%%%%%

\begin{exercice}
On note $T$ l'application linéaire définie de $L^2([0,1])$ dans lui même par $T(f)(x) = \int_{[0,x]}f(t)dt$.

\begin{enumerate}
\item
Montrer que $T^*$, l'adjoint de $T$ est donné par $T^*(f)(x) = \int_{[x,1]}f(t)dt$, et que $TT^*$ est un opérateur à noyau $K(x,y) = min\{x,y\}$.

\item
Soit $f$ un vecteur propre de $TT^{*}$ associé à la valeur propre $\lambda$. Montrer que la fonction $f$ est de classe $C^{\infty}$ (on pourra procéder par récurrence.)

\end{enumerate}

\end{exercice}


\begin{solution}

\begin{enumerate}

\item
\begin{itemize}

\item
\underline{Adjoint de T}: Soient $f,g \in H$

\begin{align}
\langle Tf,g \rangle &= \int_{[0,1]}T(f)(x)g(x)dx  \nonumber \\
&= \int_{[0,1]}g(x) (\int_{[0,x]}f(t)dt)dx  \nonumber \\
&= \int_{[0,1]} (\int_{[0,1]} f(t) 1_{t \leq x}g(x)dt)dx \nonumber
\end{align}

$| f(t) 1_{t \leq x}g(x) | \leq |f(t)g(x)|$ et comme les deux sont intégrables par le théorème de Fubini:

\begin{align}
\langle Tf,g \rangle &= \int_{[0,1]} f(t) (\int_{[0,1]} g(x) 1_{x \geq t} dx) dt \nonumber \\
&= \int_{[0,1]} f(t) (\int_{[t,1]} g(x)dx) dt \nonumber \\
&= \langle f,T^*g \rangle \nonumber
\end{align}

\item
\underline{$TT^*$ est un opérateur à noyau}:

\begin{align}
TT*(f) &= \int_{[0,x]} (\int_{[t,1]} f(y) dy) dt \nonumber \\
&= \int_{[0,1]} (\int_{[0,1]} f(y) 1_{y \geq t} 1_{x \geq t} dy) dt \nonumber \\
&= \int_{[0,1]} (\int_{[0,1]} f(y) 1_{y \geq t} 1_{x \geq t} dt) dy \nonumber \\
&= \int_{[0,1]} (\int_{[0,min(x,y)]}  f(y) dt) dy \nonumber \\
&= \int_{[0,1]} min(x,y) f(y) dy \nonumber
\end{align}

\end{itemize}

\end{enumerate}

\end{solution}

\section{Espaces de Sobolev}

%%%%%%%%%%%%%%%%%%%%

\end{document}