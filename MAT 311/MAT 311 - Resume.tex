\documentclass[10pt,a4paper,oneside]{article}

\usepackage[utf8]{inputenc}
\usepackage{amssymb}
\usepackage{amsmath}
\usepackage{fullpage}

\newtheorem{theoreme}{Théorème}
\newtheorem{proposition}{Proposition}
\newtheorem{corollaire}{Corollaire}
\newtheorem{lemme}{Lemme}

\newenvironment{definition}[1][Definition]{\begin{trivlist}
\item[\hskip \labelsep {\bfseries #1}]}{\end{trivlist}}

\newenvironment{exemple}[1][Exemple]{\begin{trivlist}
\item[\hskip \labelsep {\bfseries #1}]}{\end{trivlist}}

\begin{document}

\title{Résumé MAT311}
\author{Mario Ynocente Castro}

\maketitle

\section{Éléments de Topologie}

\begin{definition}
Une suite $(x_n)_{n \geq 0}$ d'un espace métrique $(X,d)$ est une \textbf{suite de Cauchy} si

\[ \forall \varepsilon > 0, \exists n_0 \in \mathbb{N}, \text{ tel que } (\forall n,m \geq n_0, d(x_n,x_m) < \varepsilon= \]
\end{definition}

\begin{itemize}
\item
Une suite qui converge est automatiquement une suite de Cauchy et qu'une suite de Cauchy est toujours borné.
\item
Une suite de Cauchy n'est, en général, pas convergente.
\end{itemize}

\begin{lemme}
Dans un espace métrique $(X,d)$, une suite de Cauchy $(x_n)_{n \geq 0}$ qui possède une valeur d'adhérence est convergent.
\end{lemme}

\begin{definition}
(Espace complet)
Un espace métrique $(X,d)$ est \textbf{complet} si et seulement si toute suite de Cauchy converge dans $X$.
\end{definition}

\begin{theoreme}
Un espace vectoriel normé de dimension finie, muni de la distance assoicée à la norme, est un espace métrique complet.
\end{theoreme}

\begin{definition}
Un espace de Banach est un espace vectoriel normé complet.
\end{definition}

\section{Théorie de l'intégration de Lebesgue}

\begin{theoreme}
Si $f \in \mathcal{L}^1(\Omega)$, et si

\[ \int_\Omega |f(x)| dx = 0, \]

alors $f = 0$ p.p. sur $\Omega$.
\end{theoreme}

\begin{proposition}
Soit $f \in \mathcal{L}^1(\Omega)$. Alors $|f| < +\infty$ p.p. sur $\Omega$.
\end{proposition}

\begin{theoreme}
(Théorème de la convergence monotone de Beppo Levi)
Soit $(f_n)_{n \geq 0}$ suite croissante p.p. sur $\Omega$ de fonctions de $\mathcal{L}^1(\Omega)$, telle que la suite des intégrales des fonctions $f_n$ vérifie

\[ \sup_{n \geq 0} \int_\Omega f_n(x) dx < +\infty \]

Alors, il existe $f \in \mathcal{L}^1(\Omega)$ telle que $f_n \to f$ p.p. sur $\Omega$ et

\[ \lim_{n \to \infty} \int_\Omega f_n(x) dx = \int_\Omega f(x) dx. \]
\end{theoreme}

\begin{theoreme}
(Lemme de Fatou)
Soit $(f_n)_{n \geq 0}$ une suite de fonctions de $\mathcal{L}^1(\Omega)$ telles que $f_n \geq 0$ p.p. sur $\Omega$. On suppose que la suite des intégrales des fonctions $f_n$ vérifie

\[ \sup_{n \geq 0} \int_\Omega f_n(x) dx < +\infty \]

Alors, $\underline{lim}_{n \to +\infty} f_n \in \mathcal{L}^1(\Omega)$ et

\[ \int_\Omega \underline{\lim}_{n \to +\infty} f_n(x) dx \leq \underline{\lim}_{n \to +\infty} \int_\Omega f_n(x) dx. \]
\end{theoreme}

\begin{theoreme}
(Théorème de la convergence dominée de Lebesgue)
Soit $(f_n)_{n \geq 0}$ une suite de fonctions de $\mathcal{L}^1(\Omega)$. Supposons que $f_n \to f$ p.p. sur $\Omega$ et qu'il existe $F \in \mathcal{L}^1(\Omega)$ telle que $|f_n| \leq F$ p.p. sur $\Omega$. Alors, $f \in \mathcal{L}^1(\Omega)$ et

\[ \lim_{n \to +\infty} \int_\Omega f_n(x) dx = \int_\Omega f(x) dx. \]
\end{theoreme}

\begin{definition}
Une fonction $f : \Omega \to [-\infty,+\infty]$ est mesurable s'il existe une suite $(f_n)_{n \geq 0}$ de fonctions continues à support compact qui sont définies sur $\Omega$ et qui converge vers $f$ p.p. sur $\Omega$.
\end{definition}

\begin{proposition}
La classe des fonctions mesurables sur $\Omega$ vérifie les propiétés suivantes:

\begin{enumerate}
\item
toute fonction continue sur $\Omega$ est mesurables;

\item
toute fonction continue par morceaux sur un intervalle $I \subset \mathbb{R}$ est mesurable;

\item
toute fonction appartenant à $\mathcal{L}^+(\Omega)$ est mesurable;

\item
si $f_1, \ldots, f_N$, définies sur $\Omega$, sont mesurables et si $\Phi : \mathbb{R}^N \to \mathbb{R}$ est continue, alors la fonction

\[ \Phi(f_1,\ldots,f_N) := \Phi(f_1(x),\ldots,f_N(x)), \]

est mesurable sur $\Omega$.
\end{enumerate}
\end{proposition}

\begin{corollaire}
On a les énocés suivants:

\begin{enumerate}
\item
si $f,g : \Omega \to \mathbb{R}$ sont mesurables et si $\lambda, \mu \in \mathbb{R}$, alors les fonctions $\lambda f + \mu g$ et $fg$ sont mesurables sur $\Omega$;

\item
toute fonction appartenant à $\mathcal{L}^1(\Omega)$ est mesurable sur $\Omega$;

\item
si $f,g : \Omega \to \mathbb{R}$ sont mesurables, alors les fonctions $\max(f,g)$ et $\min(f,g)$ sont mesurables. En particulier, $f^+,f^-$ et $|f|$ sont mesurables si $f$ est mesurable.
\end{enumerate}
\end{corollaire}

\begin{theoreme}
Soit $f$, fonction mesurable sur $\Omega$. Supposons qu'il existe $g \in \mathcal{L}^1(\Omega)$ telle que $|f| \leq g$ p.p. sur $\Omega$. Alors $f \in \mathcal{L}^1(\Omega)$.
\end{theoreme}

\begin{theoreme}
Soit $(f_n)_{n \geq 0}$ une suite de fonctions mesurables, définies sur $\Omega$ et $f$ une fonction définie sur $\Omega$, telles que $f_n \to f$ p.p. sur $\Omega$. Alors, la fonction $f$ est mesurable.
\end{theoreme}

\begin{corollaire}
Soit $f : \Omega \to \mathbb{R}$ une fonction mesurable telle que $f(x) \neq 0$ p.p. sur $\Omega$. Alors la fonction $x \mapsto 1/f(x)$, qui est définie p.p. sur $\Omega$, est mesurable sur $\Omega$.
\end{corollaire}

\begin{theoreme}
(Théorème de la convergence monotone de B. Levi)
Soit $(f_n)_{n \geq 0}$ unes suite de fonctions mesurables sur $\Omega$, à valeurs dans $[0,+\infty]$. Alors

\[ \int_\Omega \left( \sum_{n = 0}^\infty f_n(x) \right) dx = \sum_{n = 0}^\infty \int_\Omega f_n(x) dx \in [0,+\infty]. \]
\end{theoreme}

\begin{theoreme}
(Continuité des intégrales paramétriques)
Soient $I$ un intervalle ouvert de $\mathbb{R}$ et $\Omega$ un ouvert de $\mathbb{R}^N$, supposés non vides. Soit $f : I \times \Omega \to \mathbb{C}$ telle que, pour tout $t \in I$, la fonction $f(t,\cdot) \in \mathcal{L}^1(\Omega;\mathbb{C})$ et, pour presque tout $x \in \Omega$, la fonction $t \mapsto f(t,x)$ est continue en $t_0 \in I$. On suppose de plus qu'il existe une fonction $\Phi \in \mathcal{L}^1(\Omega)$ telle que, pour presque tout $x \in \Omega$ et pour tout $t \in I$, on ait

\[ |f(t,x)| \leq \Phi(x). \]

Alors, la fonction $F$ définie sur $I$ par

\[ F(t) := \int_\Omega f(t,x) dx, \]

est continue en $t_0$ et

\[ \lim_{t \to t_0} \int_\Omega f(t,x) dx = \int_\Omega \lim_{t \to t_0} f(t,x) dx. \]
\end{theoreme}

\begin{theoreme}
(Dérivation sous le signe somme)
Soient $I$ un intervalle ouvert de $\mathbb{R}$ et $\Omega$ un ouvert de $\mathbb{R}^N$, tous deux supposés non vides. Soit $f : I \times \Omega \to \mathbb{C}$ telle que, pour tout $t \in I$, la fonction $f(t,\cdot) \in \mathcal{L}^1(\Omega;\mathbb{C})$ et, pour presque tout $x \in \Omega$, la fonction $t \mapsto f(t,x)$ est dérivable sur $I$. On suppose de plus qu'il existe $\Phi \in \mathcal{L}^1(\Omega)$ telle que, pour presque tout $x \in \Omega$ et pour tout $t \in I$, on ait

\[ \left|\dfrac{\partial f}{\partial t}(t,x)\right| \leq \Phi(x). \]

Alors, la fonction $F$ définie sur $I$ par

\[ F(t) := \int_\Omega f(t,x) dx, \]

est dérivable sur $I$ et sa dérivée est donnée par

\[ F'(t) = \int_\Omega \dfrac{\partial f}{\partial t}(t,x) dx \]

De plus, si pour presque tout $x \in \Omega$, la fonction $f(\cdot,x) \in \mathcal{C}^1(I;\mathbb{C})$ alors la fonction $F \in \mathcal{C}^1(I;\mathbb{C})$.
\end{theoreme}

\section{Propiétés de l'intégrale de Lebesgue}

\begin{theoreme}
(Inégalité de Jensen)
Soient $f,g$ deux fonctions mesurables définies p.p. sur $\Omega$ et à valeurs réelles, et soit $\Phi : \mathbb{R} \to \mathbb{R}$ une fonction convece. Supposons que $g \geq 0$ .p. sur $\Omega$ et que

\[ \int_\Omega g(x) dx = 1. \]

Si $fg$ et $\Phi(f)g \in \mathcal{L}^1(\Omega)$, alors

\[ \Phi \left( \int_\Omega f(x)g(x) dx \right) \leq \int_\Omega \Phi(f(x))g(x) dx. \]
\end{theoreme}

\begin{theoreme}
(Inégalité de Hölder)
Soient $f,g : \Omega \to \mathbb{R}$ deux fonctions mesurables, et soient $p,q > 1$ tels que $\frac{1}{p} + \frac{1}{q} = 1$. Alors

\[ \int_\Omega |f(x) g(x)|dx \leq \left( \int_\Omega |f(x)|^p dx \right)^{1 / p} \left( \int_\Omega |g(x)|^q dx \right)^{1 / q}. \]
\end{theoreme}

\begin{lemme}
Soit $p > 1$ et $q = \frac{p}{p - 1}$, de sorte que $\frac{1}{p} + \frac{1}{q} = 1$. Alors, pour tous $X,Y > 0$, on a

\[ X^{1 / p} Y^{1/ q} \leq \frac{X}{p} + \frac{Y}{q}. \]
\end{lemme}

\begin{theoreme}
(Inégalité de Minkowski)
Soient $f, g : \Omega \to \mathbb{R}$ deux fonctions mesurables, et $p \in [1,+\infty[$. Alors

\[ \left( \int_\Omega |f(x) + g(x)|^p dx \right) \leq \left( \int_\Omega |f(x)|^p dx \right)^{1 / p} + \left( \int_\Omega |g(x)|^p dx \right)^{1 / p} \]
\end{theoreme}

\begin{theoreme}
(Théorème de Fubini)
Soient $\Omega_1 \subset \mathbb{R}^{N_1}$ et $\Omega_2 \subset \mathbb{R}^{N_2}$ deux ouverts non vides et $f \in \mathcal{L}^1(\Omega_1 \times \Omega_2)$. Alors:

\begin{enumerate}
\item
pour presque tout $x_2 \in \Omega_2$, la fonction $x_1 \mapsto f(x_1,x_2)$ appartient à $\mathcal{L}^1(\Omega_1)$;

\item
la fonction définie p.p. sur $\Omega_2$ par $x_2 \mapsto \int_{\Omega_1} f(x_1,x_2) dx_1$ appartient à $\mathcal{L}^1(\Omega_2)$;

\item
on a l'égalité

\[ \int \int_{\Omega_1 \times \Omega_2} f(x_1,x_2) dx_1 dx_2 = \int_{\Omega_1} \int_{\Omega_2} f(x_1,x_2) dx_2 dx_1 = \int_{\Omega_2} \int_{\Omega_1} f(x_1,x_2) dx_1 dx_2 \]
\end{enumerate}
\end{theoreme}

\begin{theoreme}
(Théorème de Tonelli)
\end{theoreme}

\section{Théorie de la mesure}

\begin{proposition}
Soit $\Omega$ un ouvert de $\mathbb{R}^N$ non vide et $f : \Omega \to \mathbb{R}$ une fonction définie p.p. sur $\Omega$. Alors

\begin{align*}
f\text{ mesurable} &\Leftrightarrow f^{-1}(]\lambda,+\infty[) \text{ mesurable pour tout } \lambda \in \mathbb{R}\\
&\Leftrightarrow f^{-1}([\lambda,+\infty[) \text{ mesurable pour tout } \lambda \in \mathbb{R}\\
&\Leftrightarrow f^{-1}(I) \text{ mesurable pour tout intervalle } I \subset \mathbb{R}
\end{align*}

\end{proposition}

\section{Espaces de Lebesgue}

\begin{theoreme}
(Densité de $\mathcal{C}_c(\Omega)$ dans $L^1(\Omega)$)
L'espace $\mathcal{C}_c(\Omega)$ s'identifie à un sous-espace de $L^1(\Omega)$. Autrement dit, pour tout $\varepsilon > 0$ et toute fonction $f$ intégrable sur $\Omega$, il existe $f_\varepsilon \in \mathcal{C}_c(\Omega)$ telle que

\[ \| f - f_\varepsilon \|_{L^1(\Omega)} \leq \varepsilon \]
\end{theoreme}

\begin{corollaire}
(Continuité $L^1$ des translations)
Pour toute fonction $f$ intégrable sur $\mathbb{R}^N$

\[ \lim_{|y| \to 0} \int_{\mathbb{R}^N} |f(x - y) - f(x)| dx = 0 \]
\end{corollaire}

\section{Espaces de Hilbert}

\begin{definition}
On dit qu'un sous-ensemble $C$ d'un espace vectoriel $E$ est convexe si
$\forall x,y \in C, \forall t \in [0,1], tx + (1 - t)y \in C$
\end{definition}

\begin{theoreme}
(Théorème de la projection sur un convexe fermé) Soit $H$ un espace de Hilbert et $C$ un sous-ensemble convexe fermé de $H$. Pour tout $x \in H$, il existe un unique $y \in C$ tel que

\[
\| x - y \| = d(x,C) := \inf_{z \in C} \| x - z \|
\]

\begin{itemize}

\item
Si $x \in C$ alors $y = x$

\item
Si $x \not\in C$, $y$ es caracterisé par $\Re \langle x - y,z - y \rangle \leq 0, \forall z \in C$

\end{itemize}

\end{theoreme}


\begin{theoreme}
Soit $H$ un espace de Hilbert et $F \subset H$ un sous-espace vectoriel fermé. Il existe une unique applicalition linéaire $P_F: E \to F$, telle que, pour tout $x \in H$,

\[ \| x - P_F(x) \| = d(x,F) := \inf_{z \in F} \| x - z \| \]

\begin{itemize}
\item
$P_F(x)$ est l'unique élément de $F$ vérifiant cette égalité.

\item
$x - P_F(x)$ est orthogonal à tout vecteur de F.

\item
$P_F$ est 1-Lipschitzienne (donc continue), i.e. $\| P_F(x) \| \leq \| x \|, \forall x \in E$.

\end{itemize}

\end{theoreme}


\begin{definition}
Soit $A:H \rightarrow H$ une application linéaire continue. On appelle adjoint de $A$ et on note $A^*$, une application linéaire continue $A^*:H \rightarrow H$, vérifiant

\[ \langle Ax,y \rangle = \langle x,A^*y \rangle , \forall x,y \in H \]

\end{definition}


\begin{corollaire}
Soit $A$ une application linéaire continue d'un espace de Hilbert (A est borné) dans lui-même. Alors, l'adjoint de $A$ est bien defini et c'est une application linéaire continue de $H$ dans $H$.
\end{corollaire}


\section{Tranformée de Fourier $L^2$, convergence faible et théorie spectrale}


\begin{definition}

\item
Soient $E_1$ et $E_2$ deux espaces vectoriels normés et $A: E_1 \rightarrow E_2$

\begin{itemize}

\item
On dit que A est un opérateur (linéaire) si A est une application linéaire.

\item

On dit que est un opérateur borné si A est une application linéaire, continue, i.e. s'il existe $C > 0$ telle que: $\|A(x)\|_{E_2} \leq C \|x\|_{E_1}, \forall x \in E_1$

\item
Si H est un espace de Hilbert complexe (réel), on dit que $A: H \rightarrow H$ est un opérateur hermitien (symétrique) si $A^* = A$, i.e. si

\[ <A(x),y> = <x,A(y)>, \forall x,y \in H \]

\end{itemize}

\end{definition}


\begin{definition}
Un opérateur $A: H \rightarrow H$ es compact si l'image par A de toute suite bornée admet une sous-suiste qui converge, i.e. si $(x_n)_{n\geq0}$ est une suite bornée, on peut extraire de la suite $(A(x_n))_{n\geq0}$ une sous-suite qui converge (fortement).

\begin{itemize}
\item
Un opérateur compact est borné (i.e. continu).
\end{itemize}

\end{definition}


\begin{definition}
On dit qu'une suite $(x_n)_{n\geq0}$ d'un espace d'Hilbert H converge faiblement vers x dans H si

\[ \forall y \in H, \lim\limits_{n\rightarrow+\infty}\langle x_n,y \rangle = \langle x,y \rangle \]

\begin{itemize}

\item
Notée $x_n \rightharpoonup x$

\item
Si $e_n$ est une base hilbertienne. Donc $e_n \rightharpoonup 0$.

\end{itemize}

\end{definition}


\begin{theoreme}
Soit $(x_n)_{n\geq0}$ une suite bornée dans un espace de Hilbert séparable H. Alors $(x_n)_{n\geq0}$ posséde une sous-suite $(x_{\varphi(n)})_{n\geq0}$ qui converge faiblement dans H.
\end{theoreme}


\begin{theoreme}
Si $x_n \rightharpoonup x$ alors $\| x \| \leq \varliminf_{n \to +\infty} \| x_n \|$
\end{theoreme}


\begin{proposition}
Si $x_n \rightharpoonup x$ dans $H$ et si $A: H \rightarrow H$ est un opérateur borné, alors $A(x_n) \rightharpoonup A(x)$.
\end{proposition}


\begin{definition}
On dit qu'un opérateur borné $A: H_{1} \rightarrow H_{2}$ est inversible s'il existe un opérateur borné $B: H_{2} \rightarrow H_{1}$ tel que $A \circ B = Id_{H_{2}}$ et $B \circ A = Id_{H_{1}}$
\end{definition}


\begin{definition}
Soit $A: H \rightarrow H$ un opérateur borné:

\begin{itemize}

\item
On appelle spectre de $A$  et on note $sp(A)$, l'ensemble des $\lambda \in \mathbb{C}$ tels que $A - \lambda Id_{H}$ n'est pas inversible.

\item
On appelle ensemble des valeurs propes de $A$ et on note $vp(A)$, l'ensemble des $\lambda \in \mathbb{C}$ tels que $A - \lambda Id_{H}$ n'est pas injectif, ou encore, l'ensemble des $\lambda \in \mathbb{C}$ pour lesquels il existe $x \in H - \{0\}$ tel que $A(x) = \lambda x$.

\item
$vp(A) \subset sp(A)$

\end{itemize}

\end{definition}


\begin{theoreme}
Soient H un espace de Hilbert séparable de dimension infinie et $A: H \to H$ un opérateur hermitien compact. Alors:

\begin{enumerate}
\item
Il existe $(e_n)_{n \geq 0}$ une base Hilbertienne de $H$, constituée de vecteurs propres de $A$, i.e., $A(e_n) = \lambda_n e_n$.

\item
Les valeurs propres de $A$ sont réelles et tendent vers 0.

\item
Les valeurs propres non nulles sont de multiplicité finie.

\item
$sp(A) = vp(A) \bigcup \{ 0 \}$

\end{enumerate}

\end{theoreme}


\begin{definition}
Soit $K \in L^2([0,1]^2;\mathbb{C})$. On définit l'opérateur à noyau $A_{K}$ de $L^2([0,1];\mathbb{C})$ dans lui-même, par

\[
A_{K}(f)(x) = \int_{[0,1]} K(x,y)f(y)dy
\]

\begin{itemize}

\item
$A_K$ est un opérateur compacte.

\item
L'adjoint de l'opérateur $A_{K}$ est

\[
A^{*}_{K}(f)(x) = \int_{[0,1]} \overline{K(y,x)}f(y)dy
\]

\item
$A_K$ est hermitien si et seulement si $K(x,y) = \overline{K(y,x)}$, pour presque tout $(x,y) \in [0,1]^2$.

\end{itemize}

\end{definition}

\begin{theoreme}
Si $K \in L^2([0,1]^2;\mathbb{C})$ et $A_K$ est hermitien, il existe $(\lambda_n)_{n \geq 0}$ une suite de réels qui tend vers 0 et $(\varphi_n)_{n \geq 0}$ une base hilbertienne de $L^2([0,1];\mathbb{C})$ telles que $A_K(\varphi_n) = \lambda_n \varphi_n$.

\end{theoreme}


\section{Espaces de Sobolev}


\begin{definition}
Soit $I = ]a,b[$ un intervalle ouvert, borné, non vide et $x_0 \in ]a,b[$.

L'espace de Sobolev $H^1(I;K)$ est défini par

\[ H^1(I;K):= \{ u \in L^2(I;K) : \exists v \in L^2(I;K), \exists c \in K, t.q. (*) u(x) = c + \int_{x0}^x v(t)dt \} \]

L'égalité $(*)$ doit être comprise au sens de l'égalité entre deux fonctions de $L^2(I;K)$.

\end{definition}


\begin{lemme}
Si $u \in H^1(I;K)$ alors:

\begin{enumerate}
\item
$u$ admet un représentant continu sur $]a,b[$.

\item
La décomposition $(*)$ est unique.

\item
Dans cette décomposition $c = u(x_0)$.

\end{enumerate}

La fonction v est appelée la derivée faible de u et sera notée u'.
\end{lemme}

\begin{lemme}
Muni du produit hermitien

\[ \langle u,v \rangle_{H^1(I)} := \int_{I} u \overline{v} dt + \int_{I} u' \overline{v'} dt \]

l'espace $H^1(I;K)$ est un espace de Hilbert.

\[ \| u \|_{H^1(I)} := (\int_{I} (|u(t)|^2 + |u'(t)|^2) dt)^{\frac{1}{2}} \]
\end{lemme}

\begin{definition}
On note $H_0^1(I;K) := \{ u \in H^1(I;K) : u(b) = u(a) = 0 \}$

\begin{lemme}
Muni du produit hermitien de $H^1(I;K)$, l'espace $H_0^1(I;K)$ est un espace de Hilbert.
\end{lemme}


\begin{proposition}
Les injections canoniques

\[ H^1(I;K) \to L^{\infty}(I;K) \]

et

\[ H^1(I;K) \to L^2(I;K) \]

sont compactes.
\end{proposition}


\end{definition}

\end{document}