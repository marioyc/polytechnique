\documentclass[10pt,a4paper,oneside]{article}

\usepackage[utf8]{inputenc}
\usepackage{amssymb}
\usepackage{amsmath}

\setlength\parindent{0pt}

\newenvironment{exercice}[1][Exercice]{\begin{trivlist}
\item[\hskip \labelsep {\bfseries #1}]}{\end{trivlist}}



\begin{document}

\title{PCs MAP 411}
\author{Mario Ynocente Castro}

\maketitle

\section{Introduction aux EDP et aux différences finies}

\begin{exercice}{1}

Soit $A \in \mathbb{R}^{N x N}$ une matrice symétrique positive. On considère l'équation différentielle

\begin{equation}
U'(t) = -AU(t)
\end{equation}

avec $U(0) = U_0 \in \mathbb{R}^N$. On chreche à étudier les propiétés de deux schémas aux différences finis permettant de résoudre numériquement cette équation, c.à.d. de calculer une approximation $U^n$ de $U(n \Delta t)$ où $\Delta t$ désigne le pas de discrétisation en temps.

Le premir schéma, dit schéma explicite, s'écrit

\begin{equation}
\frac{U^{n + 1} - U^n}{\Delta t} = -A U^n
\end{equation}

et le deuxième schéma, dit schéma implicite, s'écrit

\begin{equation}
\frac{U^{n + 1} - U^n}{\Delta t} = -A U^{n + 1}
\end{equation}

\textbf{Notes}

\begin{itemize}
\item
$U(t) = e^{-At}U_0$

$e^{-At} = \sum_{n = 0}^{\infty} \frac{A^n(-t)^n}{n!}$

\item
$A = P^t D P, D = diag(\lambda_1, \ldots, \lambda_n)$ et $\lambda_i > 0$ car $A$ est positive.

$V = PU \Rightarrow V'(t) = -DV(t) \Rightarrow V_i(t) = e^{\lambda_i t}V_{0,i}$
\end{itemize}

\begin{enumerate}
\item
Montrer que les deux schémas sont consistants d'ordre 1. Quelle est le schéma le plus rapide?

\textbf{Solution}

$\bar{U}^n = U(n \Delta t)$ : solution exacte au moment $n \Delta t$

\begin{align*}
\bar{U}^{n + 1} - \bar{U}^n &= U'(n \Delta t) (\Delta t) + U''(n \Delta t)\frac{(\Delta t)^2}{2}  + \mathcal{O} ((\Delta t)^3) \\
\frac{\bar{U}^{n + 1} - \bar{U}^n}{\Delta t} &= U'(n \Delta t) + \underbrace{U''(n \Delta t)}_{\| U''(n \Delta t) \| \leq \sup_{t \leq T} \| U''(t) \|}\frac{(\Delta t)}{2} + \mathcal{O} ((\Delta t)^2) \\
&= -A \bar{U}^n + \mathcal{O}(\Delta t)
\end{align*}

\begin{align*}
\bar{U}^{n + 1} - \bar{U}^n &= U'(n \Delta t + \Delta t) (\Delta t) - U''(n \Delta t + \Delta t)\frac{(\Delta t)^2}{2}  + \mathcal{O} ((\Delta t)^3) \\
\frac{\bar{U}^{n + 1} - \bar{U}^n}{\Delta t} &= U'(n \Delta t + \Delta t) - U''(n \Delta t + \Delta t)\frac{(\Delta t)}{2} + \mathcal{O} ((\Delta t)^2) \\
&= -A \bar{U}^{n + 1} + \mathcal{O}(\Delta t)
\end{align*}

\item
Montrer que la solution de $(1)$ vérifie $\| U(t) \| \leq \| U_0 \|$.

Un schéma d'approximation de l'équation $(1)$ est dit inconditionnellement stable par rapport à $A$ et $\Delta t$ s'il existe une constante $K$ telle que pour tout $A$ et tout $\Delta t > 0$,

\begin{equation}
\| U^n \| \leq K \| U^0 \| \text{ pour tout } n\Delta \leq T \text{ et tout } U^0 \in \mathbb{R}^N
\end{equation}

\item
Les schémas $(2)$ et $(3)$ sont-ils inconditionnelement stables?

\item
Montrer que le schéma explicite $(2)$ est conditionnellement stable, c'est à dire que sous une certaine condition portant sur $\Delta t$ et $A$, il existe une constante $K$ telle que $(4)$ soit vérifié.

\item
Montrer que les schémas $(2)$ et $(3)$ s'écrivent sous la forme $U^{n + 1} = B(A, \Delta t)U^n$. QUe signifie la stabilité pour la matrice $B(A, \Delta t)$?

\item
Déduire de la consistance et de la stabilité la convergence des deux schémas avec une vitesse de convergence d'ordre 1, c'est à dire que

\[ \sup_{n \Delta t < T} |U^n - U(n \Delta t)| \leq C(T,U) \Delta t \]
\end{enumerate}

\end{exercice}



\end{document}